\documentclass[a4paper,11pt,dvipsnames]{article}
\usepackage{geometry}
\geometry{
	a4paper,
	total={170mm,257mm},
	left=20mm,
	top=20mm,
}
\usepackage[linktocpage]{hyperref}
\usepackage[dvipsnames]{xcolor}

%\usepackage{graphicx}
\usepackage{cite}
\usepackage{url}
\usepackage{hyperref} % should always be the last package

\usepackage[english]{babel}
\usepackage[utf8]{inputenc}
\usepackage{fancyhdr}
\usepackage{array}

%\usepackage{pgfgantt}
%\usepackage[pdftex]{graphicx}   %Enable pdflatex
\usepackage{pgfgantt}

\newcounter{myWeekNum}
\stepcounter{myWeekNum}
%
\newcommand{\myWeek}{\themyWeekNum
	\stepcounter{myWeekNum}
	\ifnum\themyWeekNum=53
	\setcounter{myWeekNum}{1}
	\else\fi
}


\pagestyle{fancy}
\setlength{\headheight}{52pt}
\rhead{\today}

\title{Reinforcement Learning and Combinatorial Optimization}


\begin{document}
	\begin{center}	
		\textbf{\huge{DRAFT: Blockchain based decentralized reputation system.}}
		\newline
		\textsf{\large{Student: Sujata Tamang}}\\
		%\textsf{\large{Reviewer: Bjorn Victor}}\\
		%\textsf{\large{Supervisor: }}\\
		\textsf{\large{Department of Information Technology, Uppsala University}}
	\end{center}
	
	\section*{Background}
	Online identities are an essential element in the process of digital 
	interaction and requires unknown entities to trust each other based on 
	reputation system of the platform in use. Let us take a simple
	scenario to present the context. Alice wants to buy a pair of 
	headphones. To achieve this, she browses the webshop, Dbay which is 
	a buy/sell platform. She then finds a relevant ad placed by Bob who 
	has good reviews. Alice picks up her credit card, submits required 
	details and waits for her desired product to be delivered as promised. 
	In this interaction, trustworthiness of Bob is fully reliant on Dbay. 
	The entity claiming to be Bob could be Eve who found a way to bypass 
	Dbay’s security and inflate his reputation on the system. Eve could 
	delete the ad and associated account when the payment is complete or 
	she could gather Alice’s personal details to misuse it later. Any 
	malformed decision on trustworthiness of an entity could be expensive 
	and deal severe damage to the user. A centralized system is more 
	prone to directed attacks. One successful attack on Dbay can leave 
	all its users vulnerable. Thus, a central point of authority is the 
	single point of failure.\\
	On the other hand information distributed over decentralized network 
	would require simultaneous attacks on numerous accounts to achieve 
	the same effect. Added layer of cryptographic security on this 
	network would hugely raise the difficulty level of attack. In 2008, 
	a solution was proposed as Bitcoin, a peer-to-peer electronic cash 
	system \cite{Bitcoin_Satoshi} that eliminated the need for trusted central 
	authority. The underlying technology, namely blockchain is a public 
	ledger that allows anyone on the network to audit blockchains state 
	changes and prove with mathematical certainity that transactions were 
	made according to the specified blockchain rule.
	\cite{enoughBitcoinForEthereum} It comes with a computationally expensive 
	proof of work for nodes to maintain and update the blockchain database. 
	Thus, blockchain ensures fault tolerance, zero downtime, tamper-resistant 
	data. Leveraging this technology for implementing reputation system
	could be an ideal solution. The use of right reputation algorithm 
	integrated with Blockchain can ensure trustworthiness of online 
	identities with high degree of certainity.	
	
	\section*{Task}
	The task of this thesis project involves exploring current reputation 
	model and trust frameworks used in various online systems. A blockchain 
	based solution will be discussed and implemented as Proof-of-Concept for 
	computing the trust value of online identities whose reputation scores 
	are stored on public ledger for every nodes on the network to verify, 
	validate and guarantee their trustworthiness. Discussion on various 
	consensus algorithms for nodes to agree on a computational output will be
	presented. Specifically, the transaction on the network would be the transfer 
	of endorsement by known entities based on their physical or digital 
	acquaintance. Verifying acquaintance and preventing sybil
	attacks(i.e. attacker creates large number of pseudonymous identities 
	to inflate their reputation or damage someone else’s) will also be 
	discussed as part of the reputation algorithm to be used. The aim
	is to identify and investigate the problems and solutions of reputation 
	model and trust frameworks used in current online systems.
	
	\paragraph{Main goal of the thesis project: }
	 The goal is to design a PoC that can identify and protect 
	 trustworthiness of interacting entities using smart contract 
	 based decentralized reputation system.	
	
	\section*{Objective to support the main goal: }
	\begin{itemize}
		\item Definition: Literature review, state of art study, 
			identify relevant concepts.
		\item Analysis: Identify problems, solutions, reputation algorithms 
			and their use in current trust frameworks. Discuss blockchains 
			and consensus algorithms.
		\item Solution design and requirement analysis: Analyse system 
			requirements, setup test network and prerequisite environment.
		\item Implementation: Transition from design to implementation. Write 
			smart contract codes and implement functionalities discussed.
		\item Result: Measure, compare, analyze and evaluate.
		\item Documentation: This is done throughout the project timeline.
	\end{itemize}

	\section*{Method}
	Solidity, \cite{solidity} contract oriented programming language will be 
	used for writing contracts that define transactions and their exchange 
	methods on the peer-to-peer network of Ethereum. \cite{ethereum_whitepaper} 
	Test and deployment will be performed on Ethereum test network that comes 
	with preexisting test accounts. Cloud services such as DigitalOcean may 
	be used to test the validator nodes that follows discussed consensus. For 
	frontend to communicate with the contracts, web3.js can be used. Git will 
	be used as version control for the codes.
	%docker kubernetes plays well , hashicorp: terraform, nomad , Apache mesos	
	\section*{Relevant Literature}
	Relevant literatures are provided in References section as:
	\cite{literature1} \cite{literature2} \cite{ilprints562}

	\section*{Relevant Courses}
	The relevant courses that the student has undertaken at Uppsala University, 
	are listed below. They are ordered based on their relevance to the project.
	\begin{enumerate}
		\item Cryptology, 5c
		\item Secure Computer Systems, 5c
		\item Applied Cloud Computing, 10c
		\item Advanced Software Design, 5c
		\item Programming Theory, 10c
		\item Algorithms and Data Structures II, 5c
	\end{enumerate}

	\section*{Delimitations}
	Trust framework and online identities are important for any online systems. 
	This is a developing domain and plenty of research have been done with 
	discussions on limitations and future works. This project doesn’t seek to 
	overcome them but rather use the available research papers and 
	implementations as an inspiration to carry out this thesis work.
	
	\section*{Time Schedule}
	\vspace{1.5cm}
	\begin{center}
		\begin{tabular}{ |m{1.5cm}|m{30em} | } 
			\hline
			\textbf{Week} & \textbf{Descripiton}\\
			\hline
			
			5 - 7 & Literary study, related work \\ 
			8 - 11 & Investigating problem solutions \\ 
			12 - 16 & Apply solutions and write code \\
			15 & Midterm meeting with supervisor and reviewer \\ 
			17 - 19 & Measure, compare, analyze and evaluate \\ 
			20 - 21 & Write about analysis and evaluation \\
			22 - 23 & Write conclusion, discussion, complete final draft 
			for feedback \\
			24  & Prepare presentation, refactor code and documentation 
			for prototype \\
			25 & Backup time \\
			26 & Update with feedback, finalize paper, and present orally\\
			
			\hline
		\end{tabular}
	\end{center}
	
	
	
	\setcounter{myWeekNum}{5}
	\ganttset{%
		calendar week text={\myWeek{}}%
	}
	\begin{figure}[h!bt]
		\begin{center}
			\begin{ganttchart}[
				vgrid={*{6}{draw=none}, dotted},
				hgrid={*{1}{draw=none}, dotted},
				x unit=.09cm,
				y unit title=.6cm,
				y unit chart=.85cm,
				newline shortcut=true,	
				bar label node/.append style={align=justify},
				time slot format=isodate,
				time slot format/start date=2018-02-01]{2018-02-01}{2018-06-15}
				\ganttset{bar height=.6}
				\gantttitlecalendar{year, month=name, week} \\
				\ganttbar[bar/.append style={fill=BrickRed}]{Literature\ganttalignnewline Survey\ganttalignnewline }
					{2018-02-01}{2018-02-09}\\
				
				\ganttbar[bar/.append style={fill=Blue}]{\textbf{1.} Writing}
					{2018-02-05}{2018-06-10}\\
				
				\ganttbar[bar/.append style={fill=Cerulean}]{\textbf{1.1} 
				Background}
					{2018-02-12}{2018-02-23}\\
				
				\ganttbar[bar/.append style={fill=Emerald}]{\textbf{1.2} 
				Method}
					{2018-02-26}{2018-03-16}\\
				
				\ganttbar[bar/.append style={fill=SpringGreen}]{\textbf{1.3} 
				Solution Design}
					{2018-03-19}{2018-03-30}\\
				
				\ganttbar[bar/.append style={fill=Periwinkle}]{\textbf{1.4} 
				Implementation\ganttalignnewline Details}
					{2018-04-02}{2018-04-20}\\
				
				\ganttbar[bar/.append style={fill=RedViolet}]{\textbf{1.5} 
				Analysis \& \\ Evaluation}
					{2018-04-23}{2018-05-11}\\ \\
				
				\ganttbar[bar/.append style={fill=Dandelion}]{\textbf{1.6} 
				Discussion \&\\ Conclusion}
					{2018-05-14}{2018-05-25}\\
				
				\ganttbar[bar/.append style={fill=SpringGreen}]
				{Design}
					{2018-02-26}{2018-3-18}\\
				
				\ganttbar[bar/.append style={fill=Periwinkle
				}]{Implementation}{2018-03-12}{2018-04-15}\\
			
				\ganttbar[bar/.append style={fill=PineGreen}]{Testing}{2018-03-26}{2018-04-29	}\\
				
				\ganttbar[bar/.append style={fill=Violet}]{Comparison}{2018-04-15}{2018-04-29	}\\
				
				\ganttbar[bar/.append style={fill=RedViolet}]{Analysis}{2018-04-22}{2018-05-06}\\
				
				\ganttbar[bar/.append style={fill=Dandelion}]{Presentation}{2018-05-28}{2018-06-10}\\
				
			\end{ganttchart}
		\end{center}
		\caption{Time Plan}
	\end{figure}
	
	\newpage
	
	\bibliographystyle{plain}
	\bibliography{sample}
	
\end{document}

