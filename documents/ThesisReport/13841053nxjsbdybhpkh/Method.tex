%%% Ch.3: Methodology %%%
\chapter{Methodology and Implementation} \label{ch:method}
The problem of measuring the trustworthiness of communicating entities is an
essential aspect of any online system where they interact with each other for
any purpose, be it shopping, content delivery or file sharing. This chapter
follows on a discussion of a proposed endorsement network where physically or
digitally acquainted entities can endorse each other or their presented
information. The model will address several concerns such as the roles and
requirements of participants as endorser and endorsee, why a participant would
play by the rule and what is to stop them from not doing so, threat models,
etc. With a system of smart contracts, PoC design will confer interaction
between entities, aggregation of information and assignment of scores for final
computation. The storage of data both on and off-chain will be discussed.  

\section{Problem Statement}
To be able to rely on the trustworthiness of an entity as presented by any
online systems, the underlying reputation system needs to be robust and as
transparent as possible. The assurance that available information has not been
tampered with and correctness of claimed identity should be provided to sustain
minimal risk of fraud. The immutable, trustless, decentralized and distributed
attribute of blockchain protocol is a recommended solution on a public
permissionless network.

\section{ User stories \& Requirements}
Anyone can join the network and become a participant in the endorsement system.
The two notable roles of a user are endorser and endorsee. An endorser can
initiate the transaction by sending an endorsement to the participant they
trust which is accepted by the endorsee.  The same user can assume both
positions as long as a set of predefined requirements are met.  \\ 

The system requirements can be listed in points as : \\
\begin{enumerate}
	\item Allow anyone to join the network and become a member of endorsement
		system. Once registered, there must be a way for users to add profile
		information and view/edit them. 
	\item Any registered user must be allowed to send an endorsement which
		should be immediately transferred to the specified receiver's address.
		The endorser must securely sign this transaction in a way that
		identifies them such that any other participants in the network can
		verify it.
	\item All the successfully recorded endorsements should have immutable
		traceability so that anyone can go back and verify the chain of
		ownership and order in a time when it took place. Each endorsement must
		have a mapping from endorser to endorsee to trace list of endorser for
		a given endorsee and vice-versa.
	\item Allow the elimination/update of the relevant state variable if an
		endorser decides to take back the endorsement given in the past. This
		change should be evident on the network.  
	\item It must be possible for anyone within or outside the endorsement
		network to compute the total endorsement impact of any members.
	\item The public key hashes of network participants must be linked to their
		respective final score(total endorsement impact). This score must be
		publicly visible to anyone while preserving the anonymity. i.e., no
		personally identifiable information is shared publicly.
	\item Any form of an attempt to change the successfully recorded ledger
		should be evident on the network. 
\end{enumerate}

The functional requirement for each user types specifically are presented 
in the table(cite)
\begin{center}
	\begin{tabular} {| l | p{9cm} | l |}
		\hline
		\textbf{As an}  & \textbf{I need to be able to..}   & \textbf{Traceability} \\
		\hline
		\multirow{2}{*}{Endorser} &  send an endorsement to anyone I wish to as
		long as they are registered on the network so that it contributes to
		the positive impact on the system.& R2
		\\\cline{2-3} 
		& remove previously assigned endorsement so that my trust towards my
		endorsees can be reflected in real time.  & R4 \\\cline{2-3}
		& view a list of endorsees so that i can see to whom i have sent
		endorsements.& R2 \& R4 \\\cline{2-3}
		& view or edit my personal information so that i can keep it up to
		date& R1 \\\cline{2-3}
		\hline
		\multirow{2}{*}{Endorsee} & view a list of endorsers so that I can see
		from whom I have received endorsements.& R2.1 \\\cline{2-3}
		& identify me in a cryptographically secure manner so that anyone can
		verify my claim of a score on the network before the transaction. & R3 
		\\\cline{2-3}
		& view or edit my profile information so that i can keep it up to date.
		& R1 \\\cline{2-3}
		\hline
		\multirow{2}{*}{anyone} & compute the total endorsement
		impact(i.e., final computed score) of any registered participants on
		the network so that I can make an informed decision about the future
		transaction.  & R5 \\\cline{2-3}
		& join the network so that I can start sending/receiving endorsements.
		& R1 \\\cline{2-3}
		\hline
	\end{tabular}
\end{center}

The non-functional requirements are associated with the architecture and are 
of interest to the network participants: \\
\begin{itemize}
\item Security: 
\item Reliability:
\end{itemize}

\section{Component Diagram}



\section{The Model - Endorsement Network}
The initial assumption is that all nodes are honest, thereby all nodes that join 
the network are given an initial endorsement power of 1. The value is 1 to avoid 
misuse of high values later on in the game by a single node to just inflate their 
impact. This value keeps getting diluted with each interaction of given  
endorsement. This process will be further clarified later with an example.

Terminologies that will be used for this network are briefly described below : \\
\textit{nEG}: The number of connection to whom a peer has given endorsement.

\textit{nER}: The number of connection from whom a peer has received endorsement. 


\textit{Initial Endorsement Power (iep)}: This is the initial endorsement power 
granted by the network for being a participating node in the network. 
it{Endorsement power (ep):} Endorsement power is measured by how much points 
has been given by the endorser which can be determined by the number of 
connections. If a peer \textit{i} uses his \textit{iep} to give to \textit{n} 
number of peers, then the $ep_i$ is 1/n. For instance, if A endorses 
20 acquaintances, then $ep_a$ will be 1/20, if A endorses 50 
acquaintances, $ep_a$ will be 1/50 and so on. 

\textit{Endorsement impact (ei)}: Endorsement impact is not only associated with 
how much endorsement an entity has given but also with how much it has received. 
The ratio of endorsement given (EG) and of endorsement received (ER) has to be 
taken into account to create a balance in the network for each nodes. Assume the 
EG:ER is x and y respectively, let total value of received endorsement be RE , 
the ei can be calculated as \\
\begin{equation}
	ei = \frac{min(x,y)}{max(x,y)} * ep * RE 
\end{equation}
This is to ensure that EG and ER and not too far off from each other. \\ 

Received endorsement, RE is the total sum of all the endorsement received.
If a peer receives endorsement from \textit{n} peers, then the RE is given by: 

\begin{equation}
	RE = \sum_{i=0}^{n}ep_{i}
\end{equation}
%Ignore if the  received endorsement fron one connection is  1. 
%RE should be less  than the no. of connection from whom eds is received

\textit{Total endorsement impact(tei)}: The total endorsement impact determines the 
impact a node has on the network. To get a value for this, we would simply have to 
multiply the \textit{ei} with the number of connections to which they have given
endorsement. Assuming a peer \textit{i} has given endorsement to \textit{n} 
peers in the network, then the \textit{tei} would be: 
\begin{equation}
	tei = ei * n 
\end{equation}

It shows how much impact they have made on the network. This value corresponds to 
the trust score and higher the score, higher is the trustworthiness of an entity.

% With this model, there are three cases that can be expected. \\
% Case1 : EG > ER \\
 
 
 
 
% Case2 : ER > EG \\
 
 
 
 
 
 
% Case3 : EG $\simeq$ ER \\




% The idea is not to exclude malicious behavior in the network, rather include them 
% but give them no value. The endorsement power diminishes every time an entity 
% gives it to someone so the right thing to do for any entity would be to use 
% them wisely. 
% It can be analogous to a gameplay where a player is not restricted to 
% drink a life potion when his lifebar is full but he will waste it if he does with 
% no impact and when the time comes to use it, there won't be any more potion left. 

% Endorsement given (EG) \\
% Endorsement received (ER) \\

% x = 1/EG * ER \\
% y = G:R = no. of people to whom endorsement given : no. of people from whom ends. received \\

% ep = x * y \\ 

% tep = ep * (no. of connection to whom eds given) \\
% drinking potion when you have full life 
% 


%consensus algorithm - 
%what stops someone from making 250 nodes and just endorsing 
%themself.  - net flow rate convergence ,
% would detect this. as we can all 250 nodes are created just to endorse this one 
% node, and so those nodes value would soon converge to zero. 




 
%  tep determines the total impact of the node in the network, and can refer to the 
%  trustworthiness of the node in network. 
%  Cases : 
%  Case1 : EG > ER \\
%  Case2 : ER > EG \\
%  Case3 : EG ~ ER \\
 
%  The assumption is that all nodes start honest so everyone starts with 1 as iep. 
%  the ep keeps diluting as they start transfering more and more endorsements out. 
%   The goal is for the ratio to be almost close to each other such that it maintains a 
%  sort of balance. 
%  It makes no sense to give too much to a node or too less to a node so sybil attack 
%  is almost useless as creating multiple accounts just to endorse someone wont give too 
%  much value. One needs to give/send eds and build trust over time in the network. 
 
%  Problem: What happens if someone plays well in the network for a long time and in 
%  the end decides to betray the whole network. (spy)

% The goal is : once a dishonest node is detected by reputation algorithm in a 
% transaction network, all their endorsement given or received is automatically removed 
% from the network so that will change the values in the dishonest node and other nodes 
% that were connected to it in the past. 

% Question: Modeling this dynamic network and how long it would take or how efficient is it 
% to compute these values within blockchain network? 
% If it makes sense to do the computation in blockchain or off-chain. 
% Only store trust value in the bc network and do other computation elsewhere? 

% net flow rate convergence can also be used to determine anomaly in the network. 

% incentivize honest behaviour.


%Model as interaction graph
%quantify endorsement as reputation score and translate to trust value
%mention reputation algorithm as necessary to be used on the n/w. 



\section{Honest vs. Malicious Nodes}
In endorsement network, honest nodes are assumed to endorse only the
nodes on whom they have full confidence that they will perform the
fair/legitimate action on the system. In other words, they are ready
to take the risk if the identities they trusted performs malicious
activity. It takes time to build a reputation and gain enough trust value
and therefore giving it all up by endorsing malicious node would not
be a rational decision. Another assumption is that an honest node will
have a negligibly low difference between nEG and nER. On the other hand,
a malicious node will have imbalanced ration between nEG and nER. Using
the ratio of nEG and nER as one of the metrics may also alleviate the
common free rider problem discussed earlier.

%free rider problem is solved. as given and received balance
%colluding to inflate or damage
% is unanimous with the network and has 
% maximum trust score 
%profile types: progressive, consistent, fluctuating
%Definition and characteristic
%discourage malicious behaviour and encourage honest behaviour
\section{Trust Metrics}
%central nodes, 
%How is trust measured?
%What is required for it to be a correct solution? 
%objective way of accepting or rejecting the work
Every node keeps track of its neighbouring node and whenever an intera






\section{Design of PoC}
%Architecture
%various layers and their communication

Any registered user can assume both the role of endorser and endorsee. An
endorser, A must be able to join the network and start sending endorsement
right away to existing participants in the network. The only requirement for an
entity to send or receive endorsement is that they both must have joined the
network before transacting.  The maximum limit is set to 300 for a participant
to send or receive an endorsement. Based on the definition of Dunbar's number, it
is the cognitive limit to the number of people one can maintain the social
relationship with. There is nothing to stop a participant from creating
multiple identities and endorsing itself but doing so would require twice the
time which when spent on receiving or sending honest endorsement can be worth
much more.  The initial points received keeps getting replenished until the
number hits the maximum limit. The contract also allows for eliminating any
endorsement previously assigned. Thus, even when a maximum limit is reached,
users can still actively participate in the network.  Other additional
requirement includes, A node cannot self-endorse or endorse any node more than
once. 


\section{Sequence Diagram}
%optional: 



\section{Smart contracts}
%logic, functionalities, transactions, messages, 
The Endorsement contract handles the logic for sending and receiving endorsemets.  


\section{Experimental Setup} \label{sec:sectionlabel}
%Test Network describe, no. of nodes, level of difficulty etc.


\section{second section}
% It may include: Description of the methodological, theoretical, conceptual or empirical framework; design of the
% experiment; relevant steps of reasoning; data description and sources.

% Describe the approach and method(s) used to address the scientific problem. Also reflect on the particular choice of method and justify it.
