%%% Ch. 5: Discussion & Analysis%%%
\chapter{Discussion \& Analysis} \label{ch:discussion}
This chapter discusses the fulfillment of the project goal. The answers to the
research questions posed at the beginning of this project are answered
theoretically as well as implementation wise as necessary.\\

\textbf{Research question 1: }How can graph theories and relevant reputation
algorithms be used to model the interaction between entities and
detect/identify honest and malicious nodes in the network? How can the
interaction graph be modeled? \\

This is answered by section ~\ref{sec:interaction},
~\ref{sec:threatModel}.\\


\textbf{Research question 2: }What are the requirements for storing trust values
and linking them to associated identities stored off a blockchain network? How
can a blockchain application be built to define a general trust framework for a
transactional network? How could the overall system architecture look like?

This is answered by section ~\ref{sec:endorsementModel},
~\ref{ch:UserStories}, ~\ref{subsec:design_considerations}
~\ref{sec:pocDesign}. \\

\textbf{Research question 3: }How can the discussed endorsement network ensure
trustworthiness while also preserving users anonymity and how can it be
generalized to other transactional network or added on top of it to serve other
use cases such as content filtering, E-Commerce, etc.?

This is answered by section ~\ref{subsec:bcConsensus},
~\ref{sec:generalization}.\\

\section{Generalization} \label{sec:generalization}
The endorsement PoC is a general model that aggregates and assign reputation
scores to individuals based on their interaction in a network. As such, any
transactional network, e-commerce, content-serving platform, etc. can use it.
The platform-specific reputation system can still co-exist alongside the
endorsement model. It is useful to get an objective measure of the actual
transaction feedback(history, quality of services, etc.).  Consider a buy/sell
system scenario where two unknown entites, A and B are attempting to transact
with each other. The entities can check each other's rating/feedback from the
platform's reputation system in use. If that is not enough for deciding on the
transaction, they can check the endorsement network for the global reputation
score of the entity in question. As such, the decision is no more reliant only
on A's reputation or the platform's reputation. There is a third option that
guarantees a reliable and immutable data stored in a decentralized manner. From
the view of the platform, say, B/S, it allows users on endorsement network to
transact on B/S. The users of B/S gets output from decentralized trust score
storage system. The platform itself provides input to the endorsement system in
case of a failed transaction outcome to help penalize the peer in the
endorsement network. Doing so increases the accuracy in both systems. 


%The results presented in Chapter 4 are discussed and analyzed, including comments and reflections from the author. It may include the following: Comparison of obtained results with discussion,interpretation and evaluation of results. Results of analysis or modeling are described. Interpretations are drawn and connected to previous work
