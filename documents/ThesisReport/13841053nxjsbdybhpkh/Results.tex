%%% Results %%%
\chapter{Results} \label{ch:results}

%%%%%%%%%%%%%%%%%%%%%%%%%%%%%%%%%%%%%%%%%%%%%%%%%%%%%%%%%%%%%%%

%%%%%%%%%%%%%%%%%%%%%%%%%%%%%%%%%%%%%%%%%%%%%%%%%%%%%%%%%%%%%%%
\section{Interaction graph}

A participant sends an endorsement to their acquaintances. There is no absolute
way to find out if it's an honest interaction or not. i.e., if the
participating nodes are the distinct or pseudonymous identity of the same node.
The only information visible about the participants on the network are the
public address and information they chose to disclose to selected members.
Therefore, an ideal way is to view the endorsements as an interaction graph
where nodes are entities participating and edges define the interaction along
with the direction. Graph algorithms can help to determine the anamoly in the
network. Using Net flow rate convergence, anamoly detection is simplified and
explained further. 


\section{Analysis}
\section{Measurement}
\section{Comparison}
%\section{first section} \label{sec:sectionlabel}

% Presentation of results and case-study data 
% An application of the methodology is unfolded and results are presented using for example via Charts, Diagrams, Figures and Tables 
% The work is conducted in accordance with the method described earlier. Results are presented in an analytical way.
