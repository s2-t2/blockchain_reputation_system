%%% Results %%%
\chapter{Results} \label{ch:results}
This chapter evaluates the result of endorsement model as presented in section
~\ref{ch:method} by using interaction graph on existing dataset. Results
will be presented along with the discussion of measurement metrics and analyzed
to see if the requirements mentioned in section ~\ref{ch:UserStories} are met.  
%%%%%%%%%%%%%%%%%%%%%%%%%%%%%%%%%%%%%%%%%%%%%%%%%%%%%%%%%%%%%%%

%%%%%%%%%%%%%%%%%%%%%%%%%%%%%%%%%%%%%%%%%%%%%%%%%%%%%%%%%%%%%%%
\section{Interaction graph}
In order to simulate the interaction graph, existing dataset from SNAP
\cite{snapnets} was used. The dataset was extracted from Bitcoin Alpha trust
\footnote{https://alphabtc.com/blockchain/}
weighted signed network which was essentially a who-trusts-whom network of
people that trade on Bitcoin Alpha platform. Participants on this network rated
each other on a scale of -10 to +10 where negative value represented total
distrust whereas positive value represented total trust. It consists of 3,783
nodes that made 24,186 edges out of which 93\% of the edges were marked as
positive edges\cite{kumar2016edge}. 






%A participant sends an endorsement to their acquaintances. There is no absolute
%way to find out if it's an honest interaction or not. i.e., if the
%participating nodes are the distinct or pseudonymous identity of the same node.
%The only information visible about the participants on the network are the
%public address and information they chose to disclose to selected members.
%Therefore, an ideal way is to view the endorsements as an interaction graph
%where nodes are entities participating and edges define the interaction along
%with the direction. Graph algorithms can help to determine the anamoly in the
%network. Using Net flow rate convergence, anamoly detection is simplified and
%explained further. 


\section{Analysis}
\section{Measurement}
\section{Comparison}
%\section{first section} \label{sec:sectionlabel}

% Presentation of results and case-study data 
% An application of the methodology is unfolded and results are presented using for example via Charts, Diagrams, Figures and Tables 
% The work is conducted in accordance with the method described earlier. Results are presented in an analytical way.
