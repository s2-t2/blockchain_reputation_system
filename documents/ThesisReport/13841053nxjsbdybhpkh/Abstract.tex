%%% 0bAbstract/ %%%
\chapter*{Abstract}
Blockchain technology is being researched in diverse domains for its ability to
provide secure and reliable information in a distributed and decentralized
manner. Trust frameworks and reputation models of an online interaction system
are one among several use cases that require these attributes for a secure
system. The centralized nature of current online systems, however, leaves the
valuable information as such prone to both external attack and internal
modifications. This master's thesis project, therefore, studies the use of
blockchain technology as a decentralized and distributed infrastructure for an
online interaction system that can guarantee a reliable and immutable trust
score. It proposes a system of smart contracts that specify the logic for
interactions and models trust among pseudonymous identities of the system. The
contract is deployed to a blockchain network where the trust score of entities
are stored and updated in an immutable manner without any central entity. The
proposed method and the trust metrics used is evaluated by simulating an
interaction graph using an existing dataset to demonstrate the functionality,
correctness, and usability. The obtained results then illustrate how the
proposed model addresses the threat models of current reputation systems.

%Blockchain technology is being researched in diverse domains for its ability
%to provide secure and reliable information in a distributed and decentralized
%manner. Trust frameworks and reputation models of an online interaction system
%are one among several use cases that require these attributes for a secure
%system. The centralized nature of current online systems, however, leaves the
%valuable information as such prone to both external attack and internal
%modifications. This master's thesis project, therefore, intends to study the
%use of Blockchain Technology as a decentralized and distributed infrastructure
%for an online interaction system that can guarantee a reliable and immutable
%trust score. The contribution of this project is identifying the problems of
%current reputation systems, design a method to model trust among pseudonymous
%entities, collect and aggregate interaction information and assign trust score
%to individual entities. The proposed method is applied to an existing dataset,
%and an interaction graph is simulated to demonstrate the functionality,
%correctness, and usability. This project presents relevant threat models that
%exist in the current reputation system and discussion on how the proposed model
%addresses them. 


%Trust frameworks and reputation systems are essential aspects of any online
%interaction system. It helps users to evaluate the trustworthiness of
%participating entities before engaging in a transaction. Therefore, the
%information available about an online identity via a reputation system needs to
%be correct, reliable and attack resistant. Most of the online transaction
%systems have a client-server architecture with centralized data point and
%governance. Thus, the data available to them are prone to both external attack
%and internal modification. Therefore, this master's thesis intends to study the
%use of Blockchain technology as a distributed and decentralized infrastructure
%that can guarantee storage of reliable and immutable trust scores. Method of
%graph properties, network flow, and graph-based algorithms are studied and
%analyzed to motivate a solution. Use of smart contracts to define the
%interaction between network participants and their collection and aggregation
%via simple computation steps is proposed. The proposed model is applied to an
%existing real dataset to demonstrate the functionality, correctness, and
%usability. An interaction graph is presented to demonstrate the different
%behavior of users in the network. The evaluation of the interaction graph shows
%that the trust metrics are representative of user behavior in a way to be
%relevant to the definition of honest and malicious interactions. Relevant
%threat models that exist in current reputation systems and how the proposed
%model can address them is discussed. It concludes with a discussion on
%generalization and further improvements as future works. 

%Trust frameworks and reputation systems are essential aspects of any online
%interaction system. It helps users to evaluate the trustworthiness of
%participating entities before engaging in a transaction. Therefore, the
%information available about an online identity via a reputation system needs to
%be correct, reliable and attack resistant. Most of the online transaction
%systems have a client-server architecture with centralized data point and
%governance. Thus, the data available to them are prone to both external attack
%and internal modification. Therefore, this master's thesis intends to study the
%use of Blockchain technology as a distributed and decentralized infrastructure
%that can guarantee storage of reliable and immutable trust scores. Method of
%graph properties, network flow, and graph-based algorithms are studied and
%analyzed to motivate a solution. Use of smart contracts to define the
%interaction between network participants and their collection and aggregation
%via simple computation steps is proposed. The proposed model is applied to an
%existing real dataset to demonstrate the functionality, correctness, and
%usability.
%Based on the evaluation of interaction graph, results show that the
%trust metrics are representative of a trust score for the entity in question.
%The model addresses various threat models of current reputation systems and is
%supported by the results. It concludes with a discussion on generalization and
%further improvements as future works. 
