%%% 2LiteratureReview/ %%%
\chapter{Related Works}\label{ch:litrev}
The growth of online services offered possibilities to trade and interact with
anyone on a global scale freely. It also offered an accessible way for bad
actors to exploit the system, e.g., share malicious files on the network, steal
user data, or participate as a seller but not deliver the product as promised.
Thus, the advent of online interaction systems also brought along an interest
in trust and reputation management systems to formulate a method to detect and
limit possible malicious behavior in online systems. There is a significant
amount of research literature in trust frameworks and reputation management
systems. This section aims to provide an overview on some of them. First, it
refers to the literature to get the definition and requirements of a reputation
system. It then follows with a discussion on aggregation-based reputation
systems employed by current online systems. As the focus of the project is a
distributed and decentralized application, it concludes the section with a
discussion on some of the reputation algorithms for distributed systems and the
applications that use them.

\section{Reputation systems and algorithms}
As pointed out by Resnick et al.~\cite{resnick2000reputation}, a reputation
system should be able to provide enough information to help infer the
trustworthiness of participating users, encourage a user to be trustworthy and
discourage a dishonest behavior. These attributes if enforced by a reputation
system can spread honest interactions which can help to get a more accurate
trust score based on the formulation method. The definition of honest and
dishonest (or malicious) behavior depends on the context of the online
interaction system in use. For instance, an honest behavior on a P2P
file-sharing system can be based on the authenticity of the file being shared
whereas an online shopping system can base it on a successful transaction
outcome. \par 

The earliest and most known internet reputation is that of
eBay~\cite{resnick2002trust} which uses an aggregation-based reputation method
that offers a feedback-based rating system. A user in
eBay~\cite{resnick2006value} can rate a transaction along with some textual
feedback. The range of values used being \{1, 0, -1\}, positive, neutral and
negative respectively. The final aggregated score is then computed by
subtracting the total of positive and negative ratings. The
system~\cite{resnick2006value} could be judged as working based on the sales
volume and the observation that more than half the buyers usually engage in
providing feedback. However, this method fails to address issues such as Sybil
attack, inactive participation (users fear retaliation from giving negative
feedback), whitewashing attack. There are many non-eCommerce online systems
such as StackExchange, yelp, Reddit that make use of similar reputation
mechanism to filter the participating users and avoid serving malicious
participation. Most of the eCommerce systems employ a client-server
architecture which lets a central entity in control of stored data. A single
point of control is a single point of failure. In light of this, there have
been studies and proposals on decentralized reputation methods for a
distributed system. Especially significant for P2P systems such as file-sharing
or content delivery applications for detecting the quality of file or content
and the owner of those files. \par   

TrustDavis~\cite{defigueiredo2005trustdavis} is a reputation system that
addresses the concerns mentioned by Resnick et al.~\cite{resnick2002trust} by
discouraging malicious participation and incentivizing honest participation and
does so without a centralized service. It introduces the role of insurers
between interacting entities such that a user can ask to be insured for their
transaction or insure someone else transaction in exchange for a reward. If the
transaction succeeds then the buyer receives the product, and the seller gets
his/her payment as agreed upon. If the transaction fails, however, and the
buyer does not receive the product (or receives an inferior quality product)
then the buyer can contact the insurer and ask for the insured amount. The
seller can do the same if the buyer acts maliciously. The system relies on the
capability of an insurer to estimate failure probability. \par
%As such, the system recommends parties to only provide references (providing
%reference is accepting liability for a bad trade, i.e., if $A$ provides
%reference to $B$ valued at 100\$ then $A$ is willing to reimburse the buyer for
%a bad trade caused by $B$ up to the amount of 100\$) to trusted parties. 
Schaub, Alexander, et al.~\cite{schaub2016trustless} proposes a
Blockchain-based reputation model that recommends the use of blind
signature~\footnote{Blind signature schemes~\cite{katz1996handbook} is
two-party protocols between a sender $A$ and a signer $B$. $A$ sends a piece of
information to $B$ which $B$ signs and returns to $A$. From this signature, $A$
can compute $B$'s signature on an a priori message $m$ of $A$'s choice. At the
completion of the protocol, $B$ knows neither the message $m$ nor the signature
associated with it.} to disconnect the link between customer and ratings. Doing
so lets a customer freely rate/review the transaction without fear of
retaliation. The system addresses the fact that the information (shipping
address, credit card number) about a customer will need to be disclosed to the
service providers for processing the order. The system, therefore, allows a
customer to create a unique identity for each transaction. It is more
customer-centric in the sense that it allows only a customer to rate the
transaction. A merchant has no say in the quality of transactions. Allowing
only customers to rate and review a transaction leaves the possibility for a
customer to initiate an unfair rating attack towards the merchant who has no
say in it.\par

The most known and widely used~\cite{chiluka2012personalizing} reputation
algorithm in a P2P network is EigenTrust~\cite{kamvar2003eigentrust} which
recommends a method to aggregate local trust values of all peers. It uses the
notion of transitive trust, i.e., if a peer $i$ trusts a peer $j$ and peer $j$
trusts peer $k$ then $i$ would also trust $k$. Peers can rate another peer as
either positive or negative \{-1, +1\}. The users can decide if a peer can be
considered trustworthy (e.g., as a download source) based on its total
aggregated score. EigenTrust defines five issues that any P2P reputation system
should consider. They are self-policing (i.e., enforced by peers and not some
central authority), anonymity (i.e., peers' reputation should only be linked to
an opaque identifier), no profit to newcomers (i.e., reputation should be
obtained by continuous good behavior, and white-washing should not be
profitable), minimal overhead for computation/storage and robust to malicious
collectives of peers. This master's thesis project has followed these
principles closely during the solution design of the proposed model. A
significant disadvantage of EigenTrust is that it assigns a high rank to a
static set of pre-trusted peers. Pre-trusted peer is a notion of trust, where
few peers that join the network are assumed trustworthy by design.  Given the
dynamic nature of a distributed P2P networks, it is possible for a pre-trusted
peer to make a poor decision and download inauthentic files. As the algorithm
centers around pre-trusted peer, a group of nodes can form a malicious
collective and share authentic files with the pre-trusted peers to get a good
ranking while sharing malicious files with rest of the network. Thus, the
system has limited reliability in the absence of trustworthy behavior from
pre-trusted nodes~\cite{chiluka2012personalizing}.\par

%HonestPeer~\cite{kurdi2015honestpeer} proposes to enhance EigenTrust algorithm
%further by selecting reputable nodes dynamically and not just relying on
%pre-trusted nodes. It claims to have a better success rate in quality file
%serving and lower malicious participation compared to EigenTrust. \par
As mentioned by Alkharji, Sarah, et al.~\cite{alkharji2017authenticpeer++},
there are two primary methods of a reputation management systems, peer-based or
file-based reputation system. The peer-based system allows peers in the network
to be rated and assigned a reputation value. A file-based system is concerned
more with the integrity of a file that is being delivered/served on the network
regardless of who (peer) owns or serves it.
AuthenticPeer++~\cite{alkharji2017authenticpeer++} is a trust management system
for P2P networks that combines both, i.e., it shares the notion of both trusted
peers and trusted files. It only allows trusted peers to rank the file after
they have downloaded it and uses a DHT-based structure to manage the integrity
of file information. \par

Bartercast~\cite{meulpolder2009bartercast} is a distributed reputation
mechanism designed for P2P file-sharing systems. It creates a graph based on
data transfers between peers and uses the max flow algorithm to compute the
reputation values for each node. Tribler~\cite{pouwelse2008tribler} is a
BitTorrent based torrent client that uses Bartercast to rank its peers.  

PowerTrust~\cite{zhou2007powertrust} proposes a robust and scalable reputation
system that makes use of Bayesian learning. Bayesian learning is a method that
uses Bayes theorem to update the posterior probability of a hypothesis based on
prior probability and probability of the event. It uses a Bayesian method to
generate local trust scores and a distributed ranking mechanism to aggregate
reputation scores. 

\section{Trust Model}
Ries, S., Kangasharju, J. and Mühlhäuser, M.,
2006,~\cite{ries2006classification} analyzes the classification criteria of
trust systems from the aspect of the domain, dimension, and semantics of trust
values. Domains of trust values can be in binary, discrete or continuous
representation. The dimension of trust values can be either one or
multi-dimensional. In a one-dimensional approach, the value represents the
degree of trust an agent assigns to another one whereas, in a multi-dimensional
approach, the uncertainty of trust value is also allowed. The semantics of
trust values can be in the set: rating, ranking, probability, belief, and fuzzy
value. Ratings can be specified by a range of values on a scale, such as [1,4],
where 1 can represent more trustworthy, and 4 can represent less trustworthy.
Rankings are expressed in a relative way, such that a higher value means higher
trustworthiness. The probability of a trust value represents the probability
that an agent will behave as expected. Beliefs and fuzzy semantics are based on
probability theory or belief theory. Abdul-Rahman, Alfarez, and Stephen
Hailes~\cite{abdul2000supporting} mentions that trust is dynamic and
non-monotonic, i.e., additional evidence or experience with an entity can
either increase or decrease the degree of trust in another agent. They propose
a distributed trust model~\cite{abdul1998distributed} that addresses direct
trust and recommender trust. Direct trust refers to the belief of an agent in
the trustworthiness of another agent whereas recommender trust refers to the
belief of an agent in the recommendation ability of another agent. It uses
discrete labeled trust values as $\{vt, t, u, vu\}$ for representing very
trustworthy, trustworthy, untrustworthy, and very untrustworthy respectively.
Similarly, it uses $\{vg, g, b, vb\}$ for recommender trust to represent very
good, good, bad and very bad respectively. An agent in this model maintains two
separate sets for direct trust experience and recommender trust experience. PGP
trust model~\cite{abdul1997pgp} employs a web of trust approach instead of a
centralized trusted authority for deriving certainty of the owner of a public
key. Users sign each other's key that they know is authentic, and this process
helps to build a web of trust. The two areas where trust is explicit in PGP are
trustworthiness of public-key certificate and trustworthiness of introducer.
Trustworthiness of introducer refers to the ability to trust the public key in
being the signer of another public key. The degree of confidence in the
trustworthiness of public-key uses a discrete labeled value as \{undefined,
marginal, complete\}. Undefined refers to uncertainty, and marginal refers to
maybe the key is valid whereas complete refers to full confidence in that the
key is valid. Similarly, the trustworthiness of the introducer is given by
\{full, marginal, untrustworthy\}. In this case, full refers to the user's full
confidence in the key being able to introduce another public key, marginal
refers to uncertainty if the public key is fully competent, and untrustworthy
represents that the key cannot be trusted to introduce another public key.
Jøsang, Audun~\cite{josang1999algebra} proposes the trust model that uses
subjective logic to bind keys and their owners in public key infrastructure.  

\section{Blockchain applications}
The significant attributes that constitute the blockchain are distributed,
decentralized and time-stamped transactions that are stored in a
cryptographically secure manner such that they are immutable and verifiable.
There are several use cases and applications that require these attributes to
implement a secure system. Trust and reputation system being one example. As
mentioned earlier, the formulation and storage of reputation information need
to be secure enough so that participants can rely on it. Verifiable information
aids in predicting the outcome of a transaction correctly. \par 

There are several blockchain platforms in operation today that allow for the
creation of distributed and decentralized applications. Based on the use case,
they vary in design goals and principles. Bitcoin~\cite{Bitcoin_Satoshi} is
said to be the first application that made use of blockchain technology. The
purpose of bitcoin as the first application was to have a P2P electronic cash
system in an open, public, distributed, decentralized and trustless (without
the need to trust the participating nodes and without a trusted intermediary)
environment. The idea of digital cash had been around since the 1990's with
cryptographer David Chaum discussing untraceable electronic
cash~\cite{chaum1988untraceable}. However, bitcoin was able to solve the flaw
of a digital cash scheme, namely double spending problem, i.e., if $A$ had $10$
units of digital cash and sent it to $B$, then $A$ should have this amount
deducted from his account such that he is not able to spend the same amount
again. Bitcoin solved this problem without the use of trusted intermediaries
and without requiring participating entities to trust each other. Similarly, it
solved the Byzantine General's
problem~\cite{lamport1982byzantine,miller2014anonymous} probabilistically by
using Proof-Of-Work, which is a problem in distributed computing where several
agents need to agree on a state over an unreliable network and without a
trusted third party. Proof-of-work is discussed in more details by
section~\ref{subsec:consensus}. Examples of other platforms that extends
bitcoin are Namecoin~\cite{kalodner2015empirical}, which offers decentralized
name resolution system, counterparty~\footnote{https://counterparty.io/docs/},
which provides a platform for creating P2P financial applications. Hyperledger
Fabric~\cite{cachin2016architecture} is an open source permissioned distributed
ledger technology platform that allows the applications to have blockchain
components as a plug-and-play.  Everledger~\cite{lomas2015everledger} is a
blockchain platform that provides a digital, global ledger to track and protect
valuable assets. Ethereum~\cite{wood2014ethereum} is a programmable blockchain
platform that offers a Turing complete programming language and allows anyone
to write smart contracts and execute code to change the blockchain state.
Ethereum was considered to be suitable for this project for its open source
ecosystem and the availability of several test networks and active community of
developers with updated resources. Similarly, solidity~\cite{SolidityDocs} is a
smart contract language employed by various blockchain platforms such as
Ethereum~\cite{wood2014ethereum}, Tendermint~\cite{TendermintDocs},
Counterparty. Therefore, this master's thesis project uses ethereum as a
blockchain backend and solidity to write smart contracts and deploy the
application.  



%For the sake of relevance to this project, solidity as the smart
%contract language, and ethereum as the blockchain is used as a point of view
%henceforth. 
%Another blockchain based appro
%\cite{dennis2015rep} 

%To name a few of them, Sovrin(cite) is a blockchain based solution for self sovereign
%identity and decentralized trust. Wepower(cite) is another innovative use case that
%aims to tokenize the renewable energy. Cardano is a blockchain platform
%that uses a ouroboros, proof-of-stake consensus algorithm (cite). Similarly,
%policypal(cite), uport(cite) are other blockchain based application focused on
%diverse use cases.
%Leveraging blockchain technology to implement reputation system could be an ideal
%solution for measuring trustworthiness and attempt to increase accuracy of trust.
%
%move to literature review later

% Beaver(cite) is a proposed model for decentralized autonomous marketplace that utilizes zero knowledge proof for anonymizing the information of participants and thus enabling everyone to freely rate or provide feedback securely. EigenTrust is a reputation management algorithm based on the
% notion of transitive trust. i.e. If a peer i trusts a peer j, then it implies that
% i trusts all other peers trusted by j. Trustdavis(cite) makes translates the
% reputation model

%\section{Reputation Algorithms}

%\subsection{Centralized}
%%probably can move this to motivation section rather than here. 
%Currently, most of the reputation models deal with users feedback after they 
%have completed a transaction in their online platform of choice. Popular auction 
%site, ebay(footnote) is widely believed to be the biggest and best known for its
%reputation model. It uses a transaction based system where users are able to leave
%feedback about the interaction they had and can leave ratings as 0(neutral),
%1(positive) or -1(negative). These values are aggregated to compute a net score of
%users trustworthiness.
%Similarly, Q\&A websites such as Quora(footnote), Stackexchange(footnote) have their own 
%proprietary reputation model to calculate the users standings in their network. 
%Almost all the methods used are explained in web of trust(cite paper) where 
%badges, karma, activity level of users are brought into play. The common issue 
%with all of them are that they are centralized system with central point of
%authority and thus lacks transparency in correctness of information displayed.
%i.e. It is hard to tell if the information displayed about a users standing is 
%really the aggregated total of received feedback from honest users or there 
%were human intervention in the process. As it is a centralized system, it is 
%certainly possible for the admin/mod to change the rating for their 
%favored user. Other possibilities include an attacker getting access to the 
%system and changing the values to match their need. 
%
%%Since,    Also, the central point of 
%%authority implies a single point of failure making them more prone to attacks. 
%
%
%\subsection{Decentralized}
%There have also been various studies and implementations for decentralized systems 
%such as BitTorrent, gnutella network to employ an efficient reputation system. 
%Tribler(cite) uses blockchain based solution to keep track of users activity in the 
%network. Similarly, Beaver(cite) is a decentralized anonymous marketplace where 
%reputation scores and trust values are inferred from the participating users feedback 
%but they are left completely anonymous. TrustDavis(cite) presents a solution for 
%non exploitable online reputation system by defining important characteristics of 
%honest and malicious participants and incentivizing accurate ratings provided by the 
%user and discouraging dishonest behaviour.  
%\section{Existing reputation systems} \label{sec:sectionlabel}
% \subsection{Centralized}
% %ebay, yelp, feedback system
% Most of the online systems today deploy feedback system where all registered users 
% can rate or give feedback after they have made a transaction. Depending on the 
% amount of stars and positive comment, one might be assumed to be trustworthy to 
% start a transaction with. While these regular methods do provide assurance at a 
% certain level, they are simply not enough. First, they are based on centralized trust
% model of PKI which relies on certificate authority(cite).  
% %wot - remove untrustworthy middle man , blockchain - remove untrustwothy first person 
% % most popular reputation system in a centralized system is eBay,  

% \subsection{Decentralized}
% %P2P filesystem, Bittorrent, tribler, others
% EigenTrust (cite) is a reputation management algorithm for P2P system that is based on the 
% notion of transitive trust. i.e. If a peer \textit{i} trusts peer \textit{j} then \textit{i} 
% trusts all other peers trusted by \textit{j}. This method lets node generate a local trust 
% value for all the nodes it has interacted with and provides a unique global trust value for 
% the node in the network. It aims to reduce number of inauthentic files upload from untrusted 
% peers. Similarly, TrustDavis(cite) presents important distinction between honest and 
% malicious nodes behaviour in the network based on which a reputation model is proposed. 
% It makes use of max-flow network to calculate the maximum value that can flow between 
% two participating nodes. Another contribution of trustdavis is it puts insurer in the 
% middle that can vaguely be seen as escrow but have different foundation.
% Beaver, Decentralized Anonymous marketplace for doing transactions in a anonymous and 
% robust environment using bitcoin and zcash that uses zero knowledge proof. 

%\section{Problems \& Limitations}
%Existing reputation models aggregate feedbacks and evaluate actions and
%interactions of users and store them in a centralized database. i.e., A trusted
%node has the access control and rights to publish information to the network
%which implies that it could tamper with the data at will. The traditional
%client-server architecture is also susceptible to DDOS attack as the target is
%known and holds a single point of failure. Another challenge that is not
%limited to the centralized system is Sybil attack. In any digital platform that
%doesn't require one to reveal personally identifiable information, creating
%multiple pseudonymous identities to exploit the system is usually cheaper with
%nothing to lose. Sybil attack is one of the most significant challenges in a
%distributed computing environment. It is usually challenging to detect and has
%been mathematically proven to be impossible to prevent in a distributed environment.

%\subsection{Sybil Attack}
%Sybil attack is a widely used attack model in the peer-to-peer 
%reputation system. Peers in the network create multiple 
%pseudonymous identities with a purpose of inflating their 
%reputation or damaging some other peers reputation. If a peer gets
%a bad reputation in the system for its activity or other reputation
%models defined parameters, then usually it is both cheaper and  
%faster to create a new identity and start afresh then to try and 
%recuperate the damaged reputation. As the network makes it so easy
%to create identities with nothing at stake, participants opt for it 
%and exploit this feature to perform Sybil attack. 
%%gather data on attacks on current system.
% %possibilities of sybil attacks and mitigations assumed
% \subsection{Shilling Attack}
% %provides dishonest feedback
% %challenges and detection mechanism
% \subsection{Others}
% %other attacks/limitations of existing reputation models
% TASK: Literature Review covers a comprehensive presentation of the relevant scientific papers.\\

% Description of the state of the art regarding the problem/issue via scientific articles, reports, relevant publications, other data sources. Relevant literature is reviewed and forms the background to the study
