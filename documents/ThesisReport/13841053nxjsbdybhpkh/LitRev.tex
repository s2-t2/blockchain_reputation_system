%%% 2LiteratureReview/ %%%
\chapter{Literature Review}\label{ch:litrev}
The earliest and most known internet reputation is that of
eBay\footnote{https://www.ebay.com/}. It uses a feedback based rating system
where a user can rate a transaction along with some textual feedback. The range
of values used being \{1, 0, -1\}, positive, neutral and negative respectively.
The final aggregated score is computed by subtracting the total of positive and
negative ratings. This system \cite{resnick2002trust} \cite{resnick2006value}
could be judged as working based on the sales volume and the observation that
more than half the buyers usually engage in providing feedback. However, this
method fails to address issues such as Sybil attack, inactive
participation(e.g., users fear retaliation from giving negative feedback),
whitewashing etc. There are many none E-commerce online systems such as
StackExchange \footnote{https://stackexchange.com/}, Yelp
\footnote{https://www.yelp.com/}, Quora \footnote{https://www.quora.com/},
Reddit \footnote{https://www.reddit.com/} e.t.c, that make use of similar
reputation mechanism to filter the participating users and avoid serving
malicious participation. Most of the E-commerce systems employ a client-server
architecture which lets a central entity in control of stored data.  A single
point of control is a single point of failure. In light of this, there have
been several research and proposals on decentralized reputation methods for the
distributed network discussed briefly in the following sections. Especially
significant for P2P systems such as file-sharing, content delivery
applications, etc. for detecting the quality of file/content and the owner of
those files.\\
Resnick et al. \cite{resnick2000reputation}points out that a reputation system
should be able to provide enough information to help infer the trustworthiness
of participating users, encourage a user to be trustworthy and discourage
dishonest behavior. TrustDavis \cite{defigueiredo2005trustdavis} is a
reputation system that addresses these concerns. It introduces the role of
insurers between interacting entities such that a user can ask to be insured
for their transaction or insure someone else's transaction in exchange for a
reward. The system relies on the insurer's capability of estimating failure
probability. Schaub, Alexander, et al. \cite{schaub2016trustless} proposes a
block-chain based reputation model that recommends the use of blind signature
to disconnect the link between customer and ratings. Doing so lets a customer
freely rate/review the transaction without fear of retaliation. It is more
customer-centric in the sense that it allows only a customer to rate the
transaction. Thus, Sybil identities is not a concern here as a customer is
allowed to make multiple identities and in fact, a unique identifier is
recommended for a unique transaction. The obvious problem here is the unfair
rating attack. A buyer can transact with a service provider and provide negative
feedback with an intention to damage the service provider's reputation despite
their honest behavior in the network. 
%Another blockchain based appro
%\cite{dennis2015rep} 


%To name a few of them, Sovrin(cite) is a blockchain based solution for self sovereign
%identity and decentralized trust. Wepower(cite) is another innovative use case that
%aims to tokenize the renewable energy. Cardano is a blockchain platform
%that uses a ouroboros, proof-of-stake consensus algorithm (cite). Similarly,
%policypal(cite), uport(cite) are other blockchain based application focused on
%diverse use cases.
%Leveraging blockchain technology to implement reputation system could be an ideal
%solution for measuring trustworthiness and attempt to increase accuracy of trust.
%

%move to literature review later
% Beaver(cite) is a proposed model for decentralized autonomous marketplace that utilizes zero knowledge proof for anonymizing the information of participants and thus enabling everyone to freely rate or provide feedback securely. EigenTrust is a reputation management algorithm based on the
% notion of transitive trust. i.e. If a peer i trusts a peer j, then it implies that
% i trusts all other peers trusted by j. Trustdavis(cite) makes translates the
% reputation model

\section{Reputation Algorithms}
The most known and widely used reputation algorithm in a P2P network is
EigenTrust \cite{kamvar2003eigentrust} which recommends a method to aggregate
local trust values of all peers. It uses the notion of transitive trust. i.e.,
if a peer 'i' trusts a peer 'j' and peer 'j' trusts peer 'k' then 'i' would
also trust 'k'. Peers can rate another peer as either positive or negative [-1,
+1]. The users can decide if a peer can be considered trustworthy as a download
source based on its total aggregated score. EigenTrust defines five issues that
any P2P reputation system should consider. They are self-policing (i.e.,
enforced by peers and not some central authority), anonymity (i.e., peer's
reputation should only be linked to an opaque identifier), no profit to
newcomers, minimal overhead for computation/storage and robust to malicious
collectives of peers. A significant disadvantage of this algorithm is that
peers are more likely to center around a static set of pre-trusted peers that
joined the network and thus has limited reliability if they leave the network.
Pre-trusted peer is a notion of trust where few peers that join the network are
assumed trustworthy by design. Advogato's trust metric \cite{levien2003advogato}
also uses this notion.  HonestPeer \cite{kurdi2015honestpeer} proposes to
enhance EigenTrust algorithm further by selecting reputable nodes dynamically
and not just relying on pre-trusted nodes. It claims to have a better success
rate in quality file serving and lower malicious participation compared to
EigenTrust.
As mentioned by Alkharji, Sarah, et al. \cite{alkharji2017authenticpeer++}, a
reputation system can serve one of the following purpose, a peer-based
reputation system or file-based reputation system. The peer-based system allows
peers in the network to be rated and assigned a value. A file-based system is
concerned more with the integrity of a file that is being delivered/served on
the network regardless of who(peer) owns or serves it.
AuthenticPeer++ \cite{alkharji2017authenticpeer++} is a trust management system
for P2P networks that combines both. i.e., it shares the notion of both trusted
peers and trusted files. As such, it only allows trusted peers to rank the file
after they have downloaded it and uses a DHT-based structure to manage the
integrity of file information. 
Bartercast \cite{meulpolder2009bartercast} is a distributed reputation mechanism
designed for P2P file-sharing systems. It creates a graph based on data
transfers between peers and uses max flow algorithm to compute the reputation
values for each node. Tribler\footnote{https://www.tribler.org/} is a
BitTorrent based torrent client that uses Bartercast to rank its peers. 
PowerTrust \cite{zhou2007powertrust} proposes a robust and scalable reputation
system that makes use of Bayesian learning. It
uses Bayesian method to generate local trust scores and a distributed ranking
mechanism to aggregate reputation scores. 

%\subsection{Centralized}
%%probably can move this to motivation section rather than here. 
%Currently, most of the reputation models deal with users feedback after they 
%have completed a transaction in their online platform of choice. Popular auction 
%site, ebay(footnote) is widely believed to be the biggest and best known for its
%reputation model. It uses a transaction based system where users are able to leave
%feedback about the interaction they had and can leave ratings as 0(neutral),
%1(positive) or -1(negative). These values are aggregated to compute a net score of
%users trustworthiness.
%Similarly, Q\&A websites such as Quora(footnote), Stackexchange(footnote) have their own 
%proprietary reputation model to calculate the users standings in their network. 
%Almost all the methods used are explained in web of trust(cite paper) where 
%badges, karma, activity level of users are brought into play. The common issue 
%with all of them are that they are centralized system with central point of
%authority and thus lacks transparency in correctness of information displayed.
%i.e. It is hard to tell if the information displayed about a users standing is 
%really the aggregated total of received feedback from honest users or there 
%were human intervention in the process. As it is a centralized system, it is 
%certainly possible for the admin/mod to change the rating for their 
%favored user. Other possibilities include an attacker getting access to the 
%system and changing the values to match their need. 
%
%%Since,    Also, the central point of 
%%authority implies a single point of failure making them more prone to attacks. 
%
%
%\subsection{Decentralized}
%There have also been various studies and implementations for decentralized systems 
%such as BitTorrent, gnutella network to employ an efficient reputation system. 
%Tribler(cite) uses blockchain based solution to keep track of users activity in the 
%network. Similarly, Beaver(cite) is a decentralized anonymous marketplace where 
%reputation scores and trust values are inferred from the participating users feedback 
%but they are left completely anonymous. TrustDavis(cite) presents a solution for 
%non exploitable online reputation system by defining important characteristics of 
%honest and malicious participants and incentivizing accurate ratings provided by the 
%user and discouraging dishonest behaviour.  
%\section{Existing reputation systems} \label{sec:sectionlabel}
% \subsection{Centralized}
% %ebay, yelp, feedback system
% Most of the online systems today deploy feedback system where all registered users 
% can rate or give feedback after they have made a transaction. Depending on the 
% amount of stars and positive comment, one might be assumed to be trustworthy to 
% start a transaction with. While these regular methods do provide assurance at a 
% certain level, they are simply not enough. First, they are based on centralized trust
% model of PKI which relies on certificate authority(cite).  
% %wot - remove untrustworthy middle man , blockchain - remove untrustwothy first person 
% % most popular reputation system in a centralized system is eBay,  

% \subsection{Decentralized}
% %P2P filesystem, Bittorrent, tribler, others
% EigenTrust (cite) is a reputation management algorithm for P2P system that is based on the 
% notion of transitive trust. i.e. If a peer \textit{i} trusts peer \textit{j} then \textit{i} 
% trusts all other peers trusted by \textit{j}. This method lets node generate a local trust 
% value for all the nodes it has interacted with and provides a unique global trust value for 
% the node in the network. It aims to reduce number of inauthentic files upload from untrusted 
% peers. Similarly, TrustDavis(cite) presents important distinction between honest and 
% malicious nodes behaviour in the network based on which a reputation model is proposed. 
% It makes use of max-flow network to calculate the maximum value that can flow between 
% two participating nodes. Another contribution of trustdavis is it puts insurer in the 
% middle that can vaguely be seen as escrow but have different foundation.
% Beaver, Decentralized Anonymous marketplace for doing transactions in a anonymous and 
% robust environment using bitcoin and zcash that uses zero knowledge proof. 

%\section{Problems \& Limitations}
%Existing reputation models aggregate feedbacks and evaluate actions and
%interactions of users and store them in a centralized database. i.e., A trusted
%node has the access control and rights to publish information to the network
%which implies that it could tamper with the data at will. The traditional
%client-server architecture is also susceptible to DDOS attack as the target is
%known and holds a single point of failure. Another challenge that is not
%limited to the centralized system is Sybil attack. In any digital platform that
%doesn't require one to reveal personally identifiable information, creating
%multiple pseudonymous identities to exploit the system is usually cheaper with
%nothing to lose. Sybil attack is one of the most significant challenges in a
%distributed computing environment. It is usually challenging to detect and has
%been mathematically proven to be impossible to prevent in a distributed environment.

%\subsection{Sybil Attack}
%Sybil attack is a widely used attack model in the peer-to-peer 
%reputation system. Peers in the network create multiple 
%pseudonymous identities with a purpose of inflating their 
%reputation or damaging some other peers reputation. If a peer gets
%a bad reputation in the system for its activity or other reputation
%models defined parameters, then usually it is both cheaper and  
%faster to create a new identity and start afresh then to try and 
%recuperate the damaged reputation. As the network makes it so easy
%to create identities with nothing at stake, participants opt for it 
%and exploit this feature to perform Sybil attack. 
%%gather data on attacks on current system.
% %possibilities of sybil attacks and mitigations assumed
% \subsection{Shilling Attack}
% %provides dishonest feedback
% %challenges and detection mechanism
% \subsection{Others}
% %other attacks/limitations of existing reputation models
% TASK: Literature Review covers a comprehensive presentation of the relevant scientific papers.\\

% Description of the state of the art regarding the problem/issue via scientific articles, reports, relevant publications, other data sources. Relevant literature is reviewed and forms the background to the study
