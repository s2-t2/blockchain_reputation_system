%%% Ch.6: Conclusion %%%
\chapter{Conclusion} \label{ch:conclusion}
Trust system running on an interaction network can aid in a successful
interaction by pre-evaluation(i.e., evaluate the outcome of the transaction
beforehand) before actually engaging in a transaction. Collection and
aggregation of information to infer the trustworthiness of online entities are
vital for any online system. Having a reliable and trustworthy reputation
system that models users interaction helps both the reputation of the platform
in use and the honest users. The PoC designed for this project followed the
definition of a trust and reputation system closely during the implementation
phase. Specifically, encouraging honest behavior while making it difficult to
maintain a good trust score with malicious behavior. Various threat models that
concern the reputation model was addressed by game-theoretic assumptions as
well as codified restrictions. The endorsement model described in this project
can be used by any interaction model with certain modifications to meet the
requirement of the specific transaction network.\\

While the designed PoC provided a way to measure the trustworthiness of an
entity, there is no way to measure the actual trust value. The accuracy is
based on the constant feedback from the peer's interacting with an actual
transactional network. Trustworthiness is not a boolean factor that can just be
solved by a 1 or 0 for an honest or malicious peer. The task of assigning a
reputation score and inferring trustworthiness is a vague problem. The PoC,
therefore, provided a value based on several transparent factors that a user
can see for themselves and verify. Based on the evaluation of results, trust
metrics have been able to successfully represent the actual trust score of the
entity in question. From the evaluation criteria mentioned, this project has
fulfilled the goal and answered the research questions as per plan.

\section{Future Works}
Given the time constraints, the project was not able to cover all the aspects
and contexts of a reputation system. Further improvements can be made by
including the logic to directly retrieve the information from a transaction
network that can act as a feedback loop for the endorsement system. Use of
complex graph algorithms can be further implemented and evaluated to analyze
the transaction behavior on several levels.  For instance, using a
Ford-Fulkerson algorithm to compute the maximum flow on a network can be used.
Doing so can set a bound on the number of edges that an honest cluster can make
with the malicious one. Thus, detecting a malignant node to further improve on
the punishment aspects can help in the measure of more precise scores. Since
the trust and reputation are dynamic by nature, it needs to receive feedback
and update the state continually. However, use of a sophisticated algorithm
also implies a higher computation overhead.  Besides the implementation, the
project could also use more evaluation from blockchain aspects. The PoC was
deployed on Ethereum network, and transactions were mostly tested out using
ganache, which creates an in-memory nodes for generating transaction receipts
and blocks. Therefore, the speed and size of transactions, cost of
computations, etc. were taken for granted from Ethereum network. Future works
could perform a more general evaluation for the transactions and speed of the
network. \\
The primary limitation of blockchain at the current stage is its scalability
problem. Being able to handle request from a large number of users without
confirmation delays for simple transactions can help to create more adaptable
use cases in Blockchain space. It is a continually developing domain, and
research on consensus mechanisms, latency, scalability aspects of the
technology is being carried out by various sectors.





%Complex graph algorithms can be implemented to analyze the behavior on several
%levels. For instance, using a Ford-Fulkerson algorithm to compute the maximum
%flow on a network can be used. Doing so can set a bound on the number of edges
%that an honest cluster can make with the malicious one. Thus, detecting a
%malignant node to further improve on the punishment can help in the measure of
%more precise scores. Since the trust and reputation are dynamic by nature, it
%needs to receive feedback and update the state continually. However, use of
%complex algorithm also implies a higher computation overhead. The primary
%limitation of blockchain at the current stage is its scalability problem. It is
%a continually developing domain, and research by both academia and industry is
%being carried out for different use cases. Further research on consensus
%mechanism, latency, scalability issues of the technology can create several
%more adoptable use-cases. 
%
%
%%%%%%%%%%%%%%%%%%%%%%%%%%%%%%%%%%%%%%%%%%%%%%%%%%%%%%%%%%%%%%%

%%%%%%%%%%%%%%%%%%%%%%%%%%%%%%%%%%%%%%%%%%%%%%%%%%%%%%%%%%%%%%%

% Synopsis of findings, limitations, further proposals for future work on the subject. Clear conclusions are drawn that stem from the previous
%analysis. Present the conclusions drawn and the evidence and arguments
%that support the conclusions.
%
%Do not include new findings, but only refer to results already discussed in the thesis. Relevant further work in the field is summarized.
