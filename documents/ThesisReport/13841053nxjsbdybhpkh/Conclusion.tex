%%% Ch.6: Conclusion %%%
\chapter{Conclusion} \label{ch:conclusion}
While the designed PoC provided a way to measure the trustworthiness of an
entity, there is no way to measure the actual trust value. The accuracy is
based on the constant feedback from the peer's interacting with an actual
transactional network. Trustworthiness is not a boolean factor that can just be
solved by a 1 or 0 for an honest or malicious peer. The task of assigning a
reputation score and inferring trustworthiness is a vague problem. The PoC,
therefore, provided a value based on several transparent factors that a user
can see for themselves and verify. Based on the evaluation criteria mentioned,
this project has fulfilled the goal and research questions as per plan. 

\section{Future Works}
Complex graph algorithms can be implemented to analyze the behavior on several
levels. For instance, using a Ford-Fulkerson algorithm to compute the maximum
flow on a network can be used. Doing so can set a bound on the number of edges
that an honest cluster can make with the malicious one. Thus, detecting a
malignant node to further improve on the punishment can help in the measure of
more precise scores. Since the trust and reputation are dynamic by nature, it
needs to receive feedback and update the state continually. However, use of
complex algorithm also implies a higher computation overhead. The primary
limitation of blockchain at the current stage is its scalability problem. It is
a continually developing domain, and research by both academia and industry is
being carried out for different use cases. Further research on consensus
mechanism, latency, scalability issues of the technology can create several
more adoptable use-cases. 


%%%%%%%%%%%%%%%%%%%%%%%%%%%%%%%%%%%%%%%%%%%%%%%%%%%%%%%%%%%%%%%

%%%%%%%%%%%%%%%%%%%%%%%%%%%%%%%%%%%%%%%%%%%%%%%%%%%%%%%%%%%%%%%

% Synopsis of findings, limitations, further proposals for future work on the subject. Clear conclusions are drawn that stem from the previous
%analysis. Present the conclusions drawn and the evidence and arguments
%that support the conclusions.
%
%Do not include new findings, but only refer to results already discussed in the thesis. Relevant further work in the field is summarized.
