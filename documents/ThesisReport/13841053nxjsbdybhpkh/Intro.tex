%%% 1 Introduction/ %%%
\chapter{Introduction} \label{ch:intro}
%WHAT - what the reader needs to know to understand the presented 
%work, explain the background of research

\section{Definition}
\subsection{Trust and Reputation}
Trust encompasses a broad spectrum of domains and is context dependent.
Therefore, its definition varies based on context and discipline and as such
lacks collective consensus among researchers ~\cite{mcknight1996meanings} 
~\cite{mcknight2001trust}.
Using the classification from McKnight et al., 1996, Trust can be either
Personal/Interpersonal, Dispositional or Impersonal/Structural.
Personal trust is when one person trusts another specific person, persons, or
things in a particular situation. Interpersonal trust involves more than one
trusting entities. i.e., two or more people (or groups) trust each other.
Dispositional trust refers to a more general trust that is based on personality
attribute of the trusting party. It can be seen as a sense of basic
trust(attitude) and is cross-contextual. While the trust mentioned above are
implicitly directed towards a person, Impersonal/structural trust is more
likely to refer to an institutional structure such as a judiciary system. \\

Trust can be generally seen as an entity's reliance on another interacting
entity to perform a specific set of the task given a specific situation.  As
pointed out by Gambetta et al. ~\cite{gambetta2000can} ``Trust is the
subjective probability by which an agent assesses that other agent or group of
agents will perform a particular action that is beneficial or at least not
detrimental." For an entity, 'A' to trust another entity 'B' or to evaluate B's
trustworthiness, the reputation of 'B' plays a central role. Broadly defined,
Reputation is the perception of an individuals character or standing. Like
Trust, reputation is context-dependent. e.g., Alice may be trusted to answer or
use Linux questions efficiently but not Windows related questions
~\cite{zacharia2000collaborative}.  A significant difference between trust and
reputation is that the former takes the subjective measure as input whereas the
latter takes an objective standard (e.g., transaction history, ratings) as an
input to yield a resulting score that can aid in detecting
reliability/trustworthiness of an entity~\cite{Sabater2005}
~\cite{castelfranchi2000trust}. \\

The classification of trust and reputation measures based on previous survey   
 ~\cite{ josang2007survey}: 
\begin{center}
	\begin{tabularx}{\textwidth }{|X| X| X| }
		\hline
		 & Specific, vector-based & General, Synthesized \\
		 \hline
		Subjective & Survey questionnaires & eBay, voting \\
		\hline
		Objective & Product tests & Synthesised general score from product tests 
		\hline
	\end{tabularx}
	\caption{Trust and Reputation measures classification}
\end{center}
Individuals in online systems are identified by their online identities which
can be anything and not necessarily linked to their real-world identities.
Online identities play a crucial role in digital interaction and require
unknown entities to trust each other based on the reputation system of the
platform in use. As mentioned in ~\cite{rasmusson1996simulated}, trust and
reputation are soft security mechanisms where it is up to the participants
rather than the software/system to maintain security. Unlike hard security
mechanism such as access control, capabilities, authentication where a user can
be allowed or rejected access to the resource. Reputation system aids in
calculating the probability of success or risk of failure of a transaction
between interacting parties~\cite{mui2002notions}~\cite{carbone2003formal}.

%gather statistics on attacks, fraud data. 
%the trust value obtained in the endorsement network can be integrated with 
%existing reputation model of other transaction network and both combined 
%should give a higher accuracy for trustworthiness measure of entities(assumption)

\subsection{Blockchain}
Blockchain can be defined as a distributed record of state changes that let
anybody on the network audit state changes and prove with mathematical
certainty that the transactions transpired according to the blockchain
rules~\footnote{\url{https://media.consensys.net/time-sure-does-fly-ed4518792679}}.
There exist several definitions of blockchain technology each specific to their
closest use case. A formal standard definition of Blockchain is under
development as ISO/TC 307
\footnote{\url{https://www.iso.org/committee/6266604/x/catalogue/p/0/u/1/w/0/d/0}}. \\

Vitalik Buterin, the founder of Ethereum, puts it this way:  
\begin{quote}
	\centering
	"A blockchain is a magic computer that anyone can upload programs to and
	leave the programs to self-execute, where the current and all previous
	states of every program are always publicly visible, and which carries a
	very strong cryptoeconomically secured guarantee that programs running on
	the chain will continue to execute in exactly the way that the blockchain
	protocol specifies."
	\footnote{\url{https://blog.ethereum.org/2015/04/13/visions-part-1-the-value-of-blockchain-technology}} 
\end{quote}
This definition provides a broad overview of what blockchain does. As a
continually developing discipline, it keeps adapting to a new definition while
maintaining the essence. The major innovation of blockchain as an architecture
is distributed, decentralized trustless transactions~\cite{Bitcoin_Satoshi}. It
completely removed the need for an intermediary trusted third party by building
trust in the system itself. One dimension of trust as mentioned by
~\cite{miller2010trust} is trust in data which is based on stored data's
integrity. Trusting data ensures that the data is appropriate for use:
accurate, precise, available, and uncorrupted~\cite{miller2010trust}.
Blockchain achieves this by use of cryptographic schemes such as the ones
mentioned in section ~\ref{sec:cryptography} assuring tamper-resistant,
fault-tolerance, zero-downtime characteristics~\cite{swan2015blockchain}. 
\newpage

% ~\cite{enoughBitcoinForEthereum}
 
\section{Motivation}
%WHY: why is it interesting to study reputation system/trust 
% why blockchain based solution is relevant/interesting
Consider a simple scenario where Alice wants to buy a pair of headphones for
which she browses a ``buy/sell'' platform. When she finds a relevant product on the
platform published by Bob, unknown entity to Alice, the success or failure of
the transaction is dependent on two factors that may or may not be transparent.\\
(i) \textbf{Bob's reputation:} Bob's reputation can be inferred from his
history of transactions, ratings provided by previous buyers that have dealt
with him, reputation system of the platform in use, the integrity of all these
relevant data. \\
(II) \textbf{Platform's reputation:} Reputation of the platform can also be
inferred similarly based on the history of services it has been able to
provide, a general perception in the community, etc.  Here, the platform in use
acts as the trusted third party that Alice must trust to present correctly
computed, untampered data about Bob. The entity claiming to be Bob could be Eve,
who found a way to bypass the platform's security and inflate his reputation.
Eve could delete the ad and associated account when the payment is complete, or
she could gather Alice's details to misuse it later.  Any malformed decision on
the trustworthiness of an entity could be expensive and deal severe damage to
the user.\\ 
Statistics suggest that online shopping is the most adapted online
activity.\footnote{\url{https://www.experian.com/assets/decision.../reports/global-fraud-report-2018.pdf}}
Reports by Experia
\footnote{\url{https://www.experian.com/blogs/ask-experian/the-state-of-online-shopping-fraud/}}
and Javelin
\footnote{\url{https://www.javelinstrategy.com/press-release/identity-fraud-hits-all-t
ime-high-167-million-us-victims-2017-according-new-javelin}} indicate that
E-commerce fraud has risen to 30\% in 2017 from 2016 while identity fraud
victims have risen by 8\% in 2017(16.7 million U.S victims). 

Additionally, reports on fake news
\footnote{\url{https://journalistsresource.org/studies/society/internet/fake-news-conspiracy-theories-journalism-research}}$^{,}$\footnote{\url{https://www.prnewswire.com/news-releases/84-percent-of-businesses-could-reduce-fraud-risk-if-certain-about-customers-identity-300587192.html}}
that leads to spread of misinformation from malicious users or portals, attack
on an existing system continues.

A recent report on Finland's data breach exposed
\footnote{\url{https://thehackernews.com/2018/04/helsingin-uusyrityskeskus-hack.html}}
130,000 users login details while Facebook has admitted to the compromise of
2.2 billion of its user's
data\footnote{\url{https://thehackernews.com/2018/04/facebook-data-privacy.html}}.
While there are several security reasons that have led to attack at such scale.
One major reason is the client-server architecture where everything is stored
on a centralized server and data flows in and out from the same source. On the
other hand, distributing information over a decentralized network would require
a simultaneous attack to achieve the same effect, thereby increasing the
difficulty level of attack. Similarly, Reputation models can help in measuring
the reliability of interacting entities so that users can make an informed
decision before participating in any transactions. Thus, a reputation system
should be secure, robust, always available and aim for higher accuracy. The use
of right reputation algorithms with Blockchain technology could help to ensure
trustworthiness of online entities with correctness of data and a high degree
of accuracy.  

%generalize a trust framework is, therefore, a riveting problem. Graph theory
%and network flow algorithms have been researched in both centralized and
%decentralized environment before. This thesis proposes a blockchain based
%solution to record users behavior and compute a trust score for each of them. 

\section{Purpose and research questions} \label{ResearchQuestions}
%Goal of thesis
%Research questions and approach that will be taken to answer 
%them in brief.

The primary goal of this thesis is to use blockchain technology and smart
contracts to simulate an endorsement network where entities can endorse each
other based on physical or digital acquaintance. The endorsement will be
quantified to infer reputation score which in turn can yield a value that can
represent the impact the agent has made on the network.  The nodes and their
relationship will be studied to analyze honest or malicious participation.
Generalization of this endorsement network to serve other use cases shall be
discussed as well. 

The research questions that this thesis aims to address are : 
\begin{enumerate}
		\item How can graph theories and relevant reputation algorithms be used
			to model the interaction between entities and detect/identify
			honest and malicious nodes in the network? How can the interaction
			graph be modeled? \label{question1}
		\item What are the requirements for storing trust values and linking
			them to associated identities stored off a blockchain network? How
			can a blockchain application be built to define a general trust
			framework for a transactional network? How could the overall system
			architecture look like? \label{question2} 
		\item How can the discussed endorsement network ensure trustworthiness
			while also preserving users anonymity and how can it be generalized
			to other transactional network or added on top of it to serve other
			use cases such as content filtering, E-Commerce
			etc?\label{question3} 
\end{enumerate}

\section{Scope} 
%Identify goal, objective, timeplan, deliverables, 
%what the project is supposed to do and the work required to meet 
%the objective
This thesis work attempts to answer all the research questions mentioned
in section \ref{ResearchQuestions}. \\
\textbf{Research Question 1}\\
To answer research question 1, literature survey will be performed on various
reputation algorithms and existing models. This survey should follow with the
discussion on various analysis metrics and threat models eventually leading to
graph simulation of endorsement network.  

\textbf{Research Question 2 } \\
Interpretation of nodes connections and quantification of scores for individual
nodes that represents trustworthiness based on score range will be presented.
Comparative analysis of on chain vs. off-chain storage requirements will be
studied and analyzed. Overall system design and architecture will be presented. 

\textbf{Research Question 3} \\
The endorsement network will be analyzed against various network metrics to
show resilience to threat models. Discussion on other use cases and how the
endorsement model can be used on top of other systems will be presented.

%
%For research question\ref{question2}, interpretation and quantification of reputation
%scores and trust metrics will be manifested. Comparative analysis of on chain
%and off-chain storage requirements will be studied resulting in an overall
%design of endorsement system architecture. \\
%\textbf{Research Question 3}
%The endorsement network will be analysed against various network metrics to show resilience to threat models. Discussion on other use cases and how the endorsement model can be used on top of other system will be presented. 
%

%For research question\ref{question3}, relevant use cases will be presented, and the
%network will be tested on with various predefined cases and attack models to
%see how well it behaves in a dynamic environment. \\

\section{Structure of Report}
This paper is structured as follows. Chapter ~\ref{ch:litrev} will perform a
literature survey on the existing algorithms and their implementations. Chapter
~\ref{ch:background}  will provide a background overview of relevant concepts
necessary to understand the following sections. In chapter ~\ref{ch:method},
system requirements and the approach taken for the model design is shown. It
shows the overall system design and architecture.  Chapter ~\ref{ch:results}
follows on with discussion of evaluation metrics and test methods and present
results representative of the designed model. Finally, conclusion and future
works is presented in chapter ~\ref{ch:conclusion}. 

%\subsection{acronyms}
%\ac{aep}.


