%%% 1 Introduction/ %%%
\chapter{Introduction} \label{ch:intro}
%WHAT - what the reader needs to know to understand the presented 
%work, explain the background of research

\section{Definition}
\subsection{Trust and Reputation}
Trust and Reputation encompass a broad spectrum of domains and is context
dependent. Therefore, a universally agreed upon single definition doesn’t
exist. From a game theoretic sense, trust can be interpreted as a subjective
probability, by which an individual, A, expects another individual, B, to
perform a given action on which its welfare depends according to a previous
agreement. \cite{castelfranchi2000trust} Reputation, on the other hand, is the perception of an individuals
character or standing. Individuals in online systems are identified by their
online identities which can be anything and not necessarily attached to
real-world identities.\cite{Sabater2005} Online identities play a crucial role in digital
interactions and require unknown entities to trust each other based on the
reputation system of the platform in use.


%References: 
% \cite{Sabater2005}
% Trust and Control:  A dialectic link-  Cristiano Castelfranchi & Rino Falcone 

% For this reason, the central system cannot guarantee a 
% tamper-resistant data. A malformed decision on trustworthiness of an 
% entity could be expensive and deal severe damage to the user. 
% On the other hand
%gather statistics on attacks, fraud data. 
%Problems of decentralized reputation 
%trusted party will transact the trust value first to each other and those trust value 
%will be used by untrusted parties in other transaction network. 
%the trust value obtained in the endorsement network can be integrated with 
%existing reputation model of other transaction network and both combined 
%should give a higher accuracy for trustworthiness measure of entities(assumption)
%need to test when the time is right :-)


\subsection{Blockchain}
Blockchain can be defined as a distributed record of state changes that let
anybody on the network audit state changes and proves with mathematical
certainty that the transactions transpired according to the blockchain rules.
There exist several definitions of blockchain technology each specific to their
closest use case. A formal standard definition of Blockchain is under
development as ISO/TC
307.\footnote{https://www.iso.org/committee/6266604/x/catalogue/p/0/u/1/w/0/d/0}
Vitalik Buterin, the founder of Ethereum, puts it this way.  "A blockchain is a
magic computer that anyone can upload programs to and leave the programs to
self-execute, where the current and all previous states of every program are
always publicly visible, and which carries a very strong cryptoeconomically
secured guarantee that programs running on the chain will continue to execute
in exactly the way that the blockchain protocol specifies."  This definition
provides a broad overview of what blockchain
does.\footnote{https://blog.ethereum.org/2015/04/13/visions-part-1-the-value-of-blockchain-technology}
As a continually developing discipline, it needs to keep adapting to a new
definition while maintaining the essence. This thesis will discuss the topic in
more detail in the background section.
% \cite{enoughBitcoinForEthereum}
% \cite{Bitcoin_Satoshi }
% https://blog.ethereum.org/2015/04/13/visions-part-1-the-value-of-blockchain-technology
% https://www.iso.org/committee/6266604/x/catalogue/p/0/u/1/w/0/d/0

 
\section{Motivation}
%WHY: why is it interesting to study reputation system/trust 
% why blockchain based solution is relevant/interesting
Consider a simple scenario where Alice wants to buy a pair of headphones for
which she browses a buy/sell platform. When she finds a relevant product on the
platform published by Bob, unknown entity to Alice, she needs to rely on the
ratings/feedback that Bob has received on the platform from his previous
customers and also on the platform in use for not tampering with the data in
any form.  The entity claiming to be Bob could be Eve who found a way to bypass
the platform's security and inflate his reputation on the system. Eve could
delete the ad and associated account when the payment is complete, or she could
gather Alice’s personal details to misuse it later. Any malformed decision on
the trustworthiness of an entity could be expensive and deal severe damage to
the user. Thus, it is interesting to study about reputation model and methods
to make it more reliable and accurate in its measure. Reputation model offers a
way to measure the trustworthiness of entities to aid interacting users in
making an informed decision about carrying forward the transaction or dropping
it. 

Studying interactions between entities and analyzing their behavior to
generalize a trust framework is, therefore, a riveting problem. Graph theory
and network flow algorithms have been researched in both centralized and
decentralized environment before. This thesis proposes a blockchain based
solution to record users behavior and compute a trust score for each of them. 



%To name a few of them, Sovrin(cite) is a blockchain based solution for self sovereign
%identity and decentralized trust. Wepower(cite) is another innovative use case that 
%aims to tokenize the renewable energy. Cardano is a blockchain platform 
%that uses a ouroboros, proof-of-stake consensus algorithm (cite). Similarly, 
%policypal(cite), uport(cite) are other blockchain based application focused on
%diverse use cases. 
%Leveraging blockchain technology to implement reputation system could be an ideal 
%solution for measuring trustworthiness and attempt to increase accuracy of trust. 
%

%move to literature review later
% Beaver(cite) is a proposed model for decentralized autonomous marketplace that utilizes zero knowledge proof for anonymizing the information of participants and thus enabling everyone to freely rate or provide feedback securely. EigenTrust is a reputation management algorithm based on the 
% notion of transitive trust. i.e. If a peer i trusts a peer j, then it implies that 
% i trusts all other peers trusted by j. Trustdavis(cite) makes translates the 
% reputation model  

\section{Purpose and research questions} \label{ResearchQuestions}
%Goal of thesis
%Research questions and approach that will be taken to answer 
%them in brief.
The main goal of this thesis is to use blockchain technology and 
smart contracts to simulate an endorsement network where entities can 
endorse each other based on physical or digital acquaintance. The 
endorsement values will be quantified to infer reputation score and 
a trust value that can be used on any transaction network. The nodes and 
their relationship will be studied to identify honest or malicious 
participants. Generalization of this endorsement network to serve 
other use cases will also be discussed. 
The research questions that this thesis aims to address are : 
\begin{enumerate}
\item How can graph theories and relevant reputation algorithms be used 
to model the interaction between entities and detect/identify honest 
and malicious nodes in the network? How can the interaction graph be
modeled? \label{question1}
\item What are the requirements for storing trust values and linking 
them to associated identities stored off a blockchain network? How can a blockchain application be built to define a general trust framework for 
a transactional network? How could the overall system architecture look like? \label{question2} 
\item How can the discussed endorsement network ensure trustworthiness
while also preserving users anonymity and how can it be generalized to
other transactional network or added on top of it to serve other use 
cases such as content filtering, E-Commerce etc?\label{question3} 
\end{enumerate}


\section{Scope} 
%Identify goal, objective, timeplan, deliverables, 
%what the project is supposed to do and the work required to meet 
%the objective
This thesis work attempts to answer all the research questions mentioned
in section \ref{ResearchQuestions}. \\

To answer research question\ref{question1}, literature survey will be performed on
existing reputation algorithms along with the presentation of background
overview that will lead to graph simulation of endorsement network. \\ 

For research question\ref{question2}, interpretation and quantification of reputation
scores and trust metrics will be manifested. Comparative analysis of on chain
and off-chain storage requirements will be studied resulting in an overall
design of endorsement system architecture. \\

For research question\ref{question3}, relevant use cases will be presented, and the
network will be tested on with various predefined cases and attack models to
see how well it behaves in a dynamic environment. \\

\section{Structure of Report}
% define in brief(oneline) what each preceding sections are about
\subsection{acronyms}
%\ac{aep}.


