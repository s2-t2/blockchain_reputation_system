%%% Ch.3: Methodology %%%
\chapter{Methodology and Implementation} \label{ch:method}

\section{Problem Statement}
%probably defined in scope section, 
The problem of measuring the trustworthiness of communicating entities is
an essential aspect of any online system where entities interact with 
each other for any purpose, be it shopping, content delivery or file sharing. 
An endorsement network where physically or digitally acquainted entities can
endorse each others presented information is proposed. This network comes with
several questions such as who are the endorser and endorsee, why would they 
endorse the correct information, what would stop someone from just endorsing 
a random piece of information or creating multiple identities and endorsing 
themselves, etc. The storage and aggregation of information and their 
computation to generate a trust value for each participating node will be 
discussed as a blockchain based solution in the sections further. 

% \section{User stories}
% %Identify type of users/profiles
% %Write User stories : As a endorser, i can .......
% The user types in the network would be : \\
% Common entity : \\
% - should be able to present their information in the network.  \\
% - view others information that is presented publicly on the network. \\
% - be given an initial endorsement power when joining the network. \\


\section{ User stories \& Requirements}
%Functional and Non functional 
%user, storage, security requirement
%blockchain type requirement
The two primary users in the model described above is endorser and endorsee 
where endorser can initiate a transaction by giving the endorsement to an 
entity which he trusts, and endorsee can accept the transaction that was 
initiated by the endorser to receive more points in the system. 
The requirements can be listed in points as : \\
\begin{itemize}
\item Anyone should be able to join the network.  
\item There must be a way to be able to associate information presented in the 
network to the unique id of a user and assign the relevant score to it.
\item Endorser should be able to broadcast their intention of giving an 
endorsement to the endorsee and securely sign the transaction in a way that 
identifies them such that other participants in the network can verify the 
transaction. 
\item Endorser should be able to take back the endorsement that was previously 
transferred in event of which the whole chain on network that are connected 
to that particular transaction will have the previously assigned scores 
eliminated. 
\item Endorsee should be able to accept the endorsement given to them by the  
endorsers and be able to view the endorsers information that is permitted by 
the endorser. 
\item Both endorser and endorsee should be able to view the trust scores 
associated with their unique id. 
\item All the successfully recorded transactions should have an immutable 
traceability such that anyone can go back and verify the chain of ownership 
and order in time when it actually took place. 
\item Any form of attempt to change the successfully recorded ledger in block 
should be evident on the network, if not resistant. 
\end{itemize}

The functional requirement for each user types specifically are presented 
in the table(cite)


The non-functional requirements are associated with the architecture and are 
of interest to the network participants: \\
\begin{itemize}
\item Security: 
\item Reliability:
\end{itemize}

\section{Component Diagram}



\section{The Model - Endorsement Network}
The initial assumption is that all nodes are honest, thereby all nodes that join 
the network are given an initial endorsement power of 1. The value is 1 to avoid 
misuse of high values later on in the game by a single node to just inflate their 
impact. This value keeps getting diluted with each interaction of given  
endorsement. This process will be further clarified later with an example.

Terminologies that will be used for this network are briefly described below : \\
\textit{nEG}: The number of connection to whom a peer has given endorsement.

\textit{nER}: The number of connection from whom a peer has received endorsement. 


\textit{Initial Endorsement Power (iep)}: This is the initial endorsement power 
granted by the network for being a participating node in the network. 
it{Endorsement power (ep):} Endorsement power is measured by how much points 
has been given by the endorser which can be determined by the number of 
connections. If a peer \textit{i} uses his \textit{iep} to give to \textit{n} 
number of peers, then the $ep_i$ is 1/n. For instance, if A endorses 
20 acquaintances, then $ep_a$ will be 1/20, if A endorses 50 
acquaintances, $ep_a$ will be 1/50 and so on. 

\textit{Endorsement impact (ei)}: Endorsement impact is not only associated with 
how much endorsement an entity has given but also with how much it has received. 
The ratio of endorsement given (EG) and of endorsement received (ER) has to be 
taken into account to create a balance in the network for each nodes. Assume the 
EG:ER is x and y respectively, let total value of received endorsement be RE , 
the ei can be calculated as \\
\begin{equation}
	ei = \frac{min(x,y)}{max(x,y)} * ep * RE 
\end{equation}
This is to ensure that EG and ER and not too far off from each other. \\ 

Received endorsement, RE is the total sum of all the endorsement received.
If a peer receives endorsement from \textit{n} peers, then the RE is given by: 

\begin{equation}
	RE = \sum_{i=0}^{n}ep_{i}
\end{equation}
%Ignore if the  received endorsement fron one connection is  1. 
%RE should be less  than the no. of connection from whom eds is received

\textit{Total endorsement impact(tei)}: The total endorsement impact determines the 
impact a node has on the network. To get a value for this, we would simply have to 
multiply the \textit{ei} with the number of connections to which they have given
endorsement. Assuming a peer \textit{i} has given endorsement to \textit{n} 
peers in the network, then the \textit{tei} would be: 
\begin{equation}
	tei = ei * n 
\end{equation}

It shows how much impact they have made on the network. This value corresponds to 
the trust score and higher the score, higher is the trustworthiness of an entity.

% With this model, there are three cases that can be expected. \\
% Case1 : EG > ER \\
 
 
 
 
% Case2 : ER > EG \\
 
 
 
 
 
 
% Case3 : EG $\simeq$ ER \\







% The idea is not to exclude malicious behavior in the network, rather include them 
% but give them no value. The endorsement power diminishes every time an entity 
% gives it to someone so the right thing to do for any entity would be to use 
% them wisely. 
% It can be analogous to a gameplay where a player is not restricted to 
% drink a life potion when his lifebar is full but he will waste it if he does with 
% no impact and when the time comes to use it, there won't be any more potion left. 

% Endorsement given (EG) \\
% Endorsement received (ER) \\

% x = 1/EG * ER \\
% y = G:R = no. of people to whom endorsement given : no. of people from whom ends. received \\

% ep = x * y \\ 

% tep = ep * (no. of connection to whom eds given) \\
% drinking potion when you have full life 
% 


%consensus algorithm - 
%what stops someone from making 250 nodes and just endorsing 
%themself.  - net flow rate convergence ,
% would detect this. as we can all 250 nodes are created just to endorse this one 
% node, and so those nodes value would soon converge to zero. 




 
%  tep determines the total impact of the node in the network, and can refer to the 
%  trustworthiness of the node in network. 
%  Cases : 
%  Case1 : EG > ER \\
%  Case2 : ER > EG \\
%  Case3 : EG ~ ER \\
 
%  The assumption is that all nodes start honest so everyone starts with 1 as iep. 
%  the ep keeps diluting as they start transfering more and more endorsements out. 
%   The goal is for the ratio to be almost close to each other such that it maintains a 
%  sort of balance. 
%  It makes no sense to give too much to a node or too less to a node so sybil attack 
%  is almost useless as creating multiple accounts just to endorse someone wont give too 
%  much value. One needs to give/send eds and build trust over time in the network. 
 
%  Problem: What happens if someone plays well in the network for a long time and in 
%  the end decides to betray the whole network. (spy)

% The goal is : once a dishonest node is detected by reputation algorithm in a 
% transaction network, all their endorsement given or received is automatically removed 
% from the network so that will change the values in the dishonest node and other nodes 
% that were connected to it in the past. 

% Question: Modeling this dynamic network and how long it would take or how efficient is it 
% to compute these values within blockchain network? 
% If it makes sense to do the computation in blockchain or off-chain. 
% Only store trust value in the bc network and do other computation elsewhere? 

% net flow rate convergence can also be used to determine anomaly in the network. 

% incentivize honest behaviour.


%Model as interaction graph
%quantify endorsement as reputation score and translate to trust value
%mention reputation algorithm as necessary to be used on the n/w. 



\section{Honest vs. Malicious Nodes}
In endorsement network, honest nodes are assumed to endorse only the
nodes on whom they have full confidence that they will perform the
fair/legitimate action on the system. In other words, they are ready
to take the risk if the identities they trusted performs malicious
activity. It takes time to build a reputation and gain enough trust value
and therefore giving it all up by endorsing malicious node would not
be a rational decision. Another assumption is that an honest node will
have a negligibly low difference between nEG and nER. On the other hand,
a malicious node will have imbalanced ration between nEG and nER. Using
the ratio of nEG and nER as one of the metrics may also alleviate the
common free rider problem discussed earlier.

%free rider problem is solved. as given and received balance
%colluding to inflate or damage
% is unanimous with the network and has 
% maximum trust score 
%profile types: progressive, consistent, fluctuating
%Definition and characteristic
%discourage malicious behaviour and encourage honest behaviour
\section{Trust Metrics}
%central nodes, 
%How is trust measured?
%What is required for it to be a correct solution? 
%objective way of accepting or rejecting the work
\section{Token and Incentive}
%value in the network
%method to encourage good behaviour in the network
%everyone wants to have higher trust value 
%reward for validator nodes.

\section{Design of PoC}
%Architecture
%various layers and their communication
\section{Sequence Diagram}
%optional: 
\section{Smart contracts}
%logic, functionalities, transactions, messages, 
\section{Experimental Setup} \label{sec:sectionlabel}
%Test Network describe, no. of nodes, level of difficulty etc.
\section{second section}
It may include: Description of the methodological, theoretical, conceptual or empirical framework; design of the
experiment; relevant steps of reasoning; data description and sources.

Describe the approach and method(s) used to address the scientific problem. Also reflect on the particular choice of method and justify it.
