%%% 2LiteratureReview/ %%%
\chapter{Literature Review} \label{ch:litrev}
\section{Existing Reputation Systems}
\subsection{Centralized}
%probably can move this to motivation section rather than here. 
Currently, most of the reputation models deal with users feedback after they 
have completed a transaction in their online platform of choice. Popular auction 
site, ebay(footnote) is widely believed to be the biggest and best known for its
reputation model. It uses a transaction based system where users are able to leave
feedback about the interaction they had and can leave ratings as 0(neutral),
1(positive) or -1(negative). These values are aggregated to compute a net score of
users trustworthiness.
Similarly, Q\&A websites such as Quora(footnote), Stackexchange(footnote) have their own 
proprietary reputation model to calculate the users standings in their network. 
Almost all the methods used are explained in web of trust(cite paper) where 
badges, karma, activity level of users are brought into play. The common issue 
with all of them are that they are centralized system with central point of
authority and thus lacks transparency in correctness of information displayed.
i.e. It is hard to tell if the information displayed about a users standing is 
really the aggregated total of received feedback from honest users or there 
were human intervention in the process. As it is a centralized system, it is 
certainly possible for the admin/mod to change the rating for their 
favored user. Other possibilities include an attacker getting access to the 
system and changing the values to match their need. 

%Since,    Also, the central point of 
%authority implies a single point of failure making them more prone to attacks. 


\subsection{Decentralized}
There have also been various studies and implementations for decentralized systems 
such as BitTorrent, gnutella network to employ an efficient reputation system. 
Tribler(cite) uses blockchain based solution to keep track of users activity in the 
network. Similarly, Beaver(cite) is a decentralized anonymous marketplace where 
reputation scores and trust values are inferred from the participating users feedback 
but they are left completely anonymous. TrustDavis(cite) presents a solution for 
non exploitable online reputation system by defining important characteristics of 
honest and malicious participants and incentivizing accurate ratings provided by the 
user and discouraging dishonest behaviour.  
%\section{Existing reputation systems} \label{sec:sectionlabel}
% \subsection{Centralized}
% %ebay, yelp, feedback system
% Most of the online systems today deploy feedback system where all registered users 
% can rate or give feedback after they have made a transaction. Depending on the 
% amount of stars and positive comment, one might be assumed to be trustworthy to 
% start a transaction with. While these regular methods do provide assurance at a 
% certain level, they are simply not enough. First, they are based on centralized trust
% model of PKI which relies on certificate authority(cite).  
% %wot - remove untrustworthy middle man , blockchain - remove untrustwothy first person 
% % most popular reputation system in a centralized system is eBay,  

% \subsection{Decentralized}
% %P2P filesystem, Bittorrent, tribler, others
% EigenTrust (cite) is a reputation management algorithm for P2P system that is based on the 
% notion of transitive trust. i.e. If a peer \textit{i} trusts peer \textit{j} then \textit{i} 
% trusts all other peers trusted by \textit{j}. This method lets node generate a local trust 
% value for all the nodes it has interacted with and provides a unique global trust value for 
% the node in the network. It aims to reduce number of inauthentic files upload from untrusted 
% peers. Similarly, TrustDavis(cite) presents important distinction between honest and 
% malicious nodes behaviour in the network based on which a reputation model is proposed. 
% It makes use of max-flow network to calculate the maximum value that can flow between 
% two participating nodes. Another contribution of trustdavis is it puts insurer in the 
% middle that can vaguely be seen as escrow but have different foundation.
% Beaver, Decentralized Anonymous marketplace for doing transactions in a anonymous and 
% robust environment using bitcoin and zcash that uses zero knowledge proof. 

\section{Problems \& Limitations}
The major problem with centralized system is their central point of authority. Therefore, 
it is hard to always trust the ratings shown there. Since, a human is in control of 
the database, they might change it to meet their need or an attacker can gain access 
to it and manipulate the system to meet their need. Therefore, they cannot assure a 
tamper resistant, always available and verifiable data with their architecture. 
The problem with decentralized system is Sybil attack where a user creates multiple 
pseudonymous identities to exploit the system and inflate his reputation. This is 
usually difficult to detect and have been proven to be practically impossible to 
prevent in a distributed computing environment (cite). Another issue is shilling attack 
where a user intentionally provides dishonest feedback for a transaction. There is also 
a possibility for multiple users to collude and manipulate the system to inflate the 
group members reputation. 
%There have been several researches done to resist these 
% attack and are on going.
%need to write more 

\subsection{Sybil Attack}
Sybil attack is a widely used attack model in the peer-to-peer 
reputation system. Peers in the network create multiple 
pseudonymous identities with a purpose of inflating their 
reputation or damaging some other peers reputation. If a peer gets
a bad reputation in the system for its activity or other reputation
models defined parameters, then usually it is both cheaper and  
faster to create a new identity and start afresh then to try and 
recuperate the damaged reputation. As the network makes it so easy
to create identities with nothing at stake, participants opt for it 
and exploit this feature to perform Sybil attack. 
%%gather data on attacks on current system.
% %possibilities of sybil attacks and mitigations assumed
% \subsection{Shilling Attack}
% %provides dishonest feedback
% %challenges and detection mechanism
% \subsection{Others}
% %other attacks/limitations of existing reputation models
% TASK: Literature Review covers a comprehensive presentation of the relevant scientific papers.\\

% Description of the state of the art regarding the problem/issue via scientific articles, reports, relevant publications, other data sources. Relevant literature is reviewed and forms the background to the study
