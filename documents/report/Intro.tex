%%% 1 Introduction/ %%%
\chapter{Introduction} \label{ch:intro}
%WHAT - what the reader needs to know to understand the presented 
%work, explain the background of research

\section{Definition}
\subsection{Trust and Reputation}
Trust and Reputation encompass a wide spectrum of domains and is
context dependent. Therefore, a single definition that is universally
agreed upon doesn't exist. If we take the game theoretic approach of
viewing them, trust can be defined as a subjective probability, by
which an individual, A expects another individual, B, to perform
%\todo{to perform} a given
a given action on which its welfare depends according to a previous 
agreement.
%\todo{according to a previous agreement}.
Similarly, reputation can be 
defined as the perception of an individuals character or standing
\cite{Sabater2005}. Individuals in online systems are identified by
their online identities which can be anything and not necessarily
attached to real-world identities. Online identities play a crucial
role in digital interactions and require unknown entities to trust each
other based on the reputation system of the platform in use. The 
client-server architecture of most online systems makes it easy to 
attack them. For instance, performing a DOS attack would require
attacking one central server and all the users whose information lives
on that server will lose access during the time of the attack. The 
architecture with a central point of authority has a centralized entity
in control of the database. While users may be able to read or modify
data, it is the central authority that one needs to rely on for
providing trustworthy information enough to make a good decision on the
network.  
% For this reason, the central system cannot guarantee a 
% tamper-resistant data. A malformed decision on trustworthiness of an 
% entity could be expensive and deal severe damage to the user. 
% On the other hand
%gather statistics on attacks, fraud data. 
%Problems of decentralized reputation 
%trusted party will transact the trust value first to each other and those trust value 
%will be used by untrusted parties in other transaction network. 
%the trust value obtained in the endorsement network can be integrated with 
%existing reputation model of other transaction network and both combined 
%should give a higher accuracy for trustworthiness measure of entities(assumption)
%need to test when the time is right :-)


\subsection{Blockchain}
% As the name suggests, Blockchain is a chain of blocks where each block 
% is linked to the previous block by cryptographic hash and can be seen 
% as a singly linked list. Thus, modifying information on a block would 
% require one to go all the way back 
Blockchain can be defined as a distributed 
%public\todo{Really public? Not always.} 
record of state changes that lets 
anybody on the network audit state changes and prove with 
mathematical certainty that the transactions were made according 
to the blockchain rules \cite{enoughBitcoinForEthereum}. Each block consists 
of lists of validated transactions and are linked by cryptographic hash 
to previous block. Thus, it is practically impossible to modify a block 
without being evident on the network. This technology came into being 
in 2008 as Bitcoin, a peer-to-peer electronic cash system, 
\cite{Bitcoin_Satoshi} which allowed unknown and not necessarily trusted 
entities to transfer value to each other without the need of any central 
trusted third party. Blockchain can be permissionless or permissioned, public or 
private depending on the application and use cases. Different applications may 
employ different consensus algorithms to maintain and update the blockchain 
database. Blockchain ensures fault tolerant, zero downtime, and tamper resistant 
data to be stored on its network. 
%more stuffs here
 
\section{Motivation}
%WHY: why is it interesting to study reputation system/trust 
% why blockchain based solution is relevant/interesting
Any form of digital interaction that takes place online requires participating 
entities to trust each other based on reputation system of platform in use. 
Let us consider a simple scenario where Alice wants to buy a pair of headphones for 
which she browses a buy/sell platform. She then finds a relevant ad placed by Bob 
who has good reviews and rating in the respective platform. Alice picks up her 
credit card, submits required details and waits for her desired product to be  
delivered as promised. In this interaction, trustworthiness of Bob is fully 
reliant on the reputation/rating system of the platform and the correctness of 
information stored on their database. The entity claiming to 
be Bob could be Eve who found a way to bypass the platform's security and inflate 
his reputation on the system. Eve could delete the ad and associated account 
when the payment is complete or she could gather Alice's personal details to 
misuse it later. Any malformed decision on trustworthiness of an entity could be 
expensive and deal severe damage to the user. Thus, it is interesting to study about 
reputation systems and methods to make it more reliable and accurate in its measure.
We can say reputation system offers a way to measure trustworthiness of entities to 
aid interacting entities in making an informed decision about carrying forward 
the transaction or dropping it. 
%\todo{maybe this is too early to focus only on transactions, it's not clear why that is }. 
Studying interactions between entities and analyzing their behaviour to generalize 
a trust framework is therefore a riveting problem. Graph theory and network flow
algorithms have been used in the past (cite) 
to study interaction between nodes and prevent any possible malicious 
behaviour in the network.
Blockchain is a fairly new technology and various use cases are being proposed and developed. 
%\todo{insert examples.} 
To name a few of them, Sovrin(cite) is a blockchain based solution for self sovereign
identity and decentralized trust. Wepower(cite) is another innovative use case that 
aims to tokenize the renewable energy. Cardano is a blockchain platform 
that uses a ouroboros, proof-of-stake consensus algorithm (cite). Similarly, 
policypal(cite), uport(cite) are other blockchain based application focused on
diverse use cases. 
Leveraging blockchain technology to implement reputation system could be an ideal 
solution for measuring trustworthiness and attempt to increase accuracy of trust. 


%move to literature review later
% Beaver(cite) is a proposed model for decentralized autonomous marketplace that utilizes zero knowledge proof for anonymizing the information of participants and thus enabling everyone to freely rate or provide feedback securely. EigenTrust is a reputation management algorithm based on the 
% notion of transitive trust. i.e. If a peer i trusts a peer j, then it implies that 
% i trusts all other peers trusted by j. Trustdavis(cite) makes translates the 
% reputation model  

\section{Purpose and research questions}
%Goal of thesis
%Research questions and approach that will be taken to answer 
%them in brief.
The main goal of this thesis is to use blockchain technology and 
smart contracts to simulate an endorsement network where entities can 
endorse each other based on physical or digital acquaintance. The 
endorsement values will be quantified to infer reputation score and 
a trust value that can be used on any transaction network. The nodes and 
their relationship will be studied to identify honest or malicious 
participants. Generalization of this endorsement network to serve 
other use cases will also be discussed. 
The research questions that this thesis aims to address are : 
\begin{enumerate}
\item How can graph theories and relevant reputation algorithms be used 
to model the interaction between entities and detect/identify honest 
and malicious nodes in the network? How can the interaction graph be
modeled?
\item What are the requirements for storing trust values and linking 
them to associated identities stored off a blockchain network? How can a blockchain application be built to define a general trust framework for 
a transactional network? How could the overall system architecture look like? 
\item How can the discussed endorsement network ensure trustworthiness
while also preserving users anonymity and how can it be generalized to
other transactional network or added on top of it to serve other use 
cases such as content filtering, E-Commerce etc? 
\end{enumerate}


\section{Scope} 
%Identify goal, objective, timeplan, deliverables, 
%what the project is supposed to do and the work required to meet 
%the objective
This thesis work attempts to answer all the research questions mentioned
in section(refer section).
 
For Research question1, literature survey will be performed on various 
reputation algorithms in use and a background overview will be 
performed that will lead to graph simulation of endorsement network.  
For Research question2, methods to quantify reputation scores and trust 
values will be devised. Blockchain storage and off blockchain storage 
such as ipfs will be studied and both methods will be compared and 
analysed. The system architecture for proposed trust framework will be 
presented. For Research question3, Several use cases will be presented 
and the network will be tested on with various predefined cases to see 
how well it acts in the dynamic network. 

\section{Structure of Report}
% define in brief(oneline) what each preceding sections are about
\subsection{acronyms}
%\ac{aep}.


