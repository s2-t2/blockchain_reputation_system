\documentclass[a4paper,11pt,dvipsnames]{article}
\usepackage{geometry}
\geometry{
	a4paper,
	total={170mm,257mm},
	left=20mm,
	top=5mm,
}
\usepackage[linktocpage]{hyperref}
\usepackage[dvipsnames]{xcolor}

%\usepackage{graphicx}
\usepackage{cite}
\usepackage{url}
\usepackage{hyperref} % should always be the last package

\usepackage[english]{babel}
\usepackage[utf8]{inputenc}
\usepackage{fancyhdr}
\usepackage{array}

%\usepackage{pgfgantt}
\usepackage[pdftex]{graphicx}   %Enable pdflatex
\usepackage{pgfgantt}

\newcounter{myWeekNum}
\stepcounter{myWeekNum}
%
\newcommand{\myWeek}{\themyWeekNum
	\stepcounter{myWeekNum}
	\ifnum\themyWeekNum=53
	\setcounter{myWeekNum}{1}
	\else\fi
}


\pagestyle{fancy}
\setlength{\headheight}{52pt}
\rhead{\today}

%\title{Blockchain and Smart Contracts }
%Decentralized Reputation Model and General Trust Framework

\begin{document}
	\begin{center}
		\textbf{\huge{Blockchain based Decentralized Reputation Model and 
		General Trust Framework}}
		\newline
%		\large Blockchain and SmartContract. \\
		\textsf{\large{Student: Sujata Tamang}}\\
		\textsf{\large{Supervisor: Jonatan H. Bergquist }}\\
		\textsf{\large{Reviewer: Björn Victor}}\\
		\textsf{\large{Department of Information Technology, Uppsala University}}
	\end{center}
	
	\section*{Background}
%	In any online application, interaction between participating entities 
%	requires them to trust each other based on the reputation system of 
%	respective platform in use. During a buy/sell interaction, both buyer 
%	and seller needs to be able to rely on one another for ensuring a 
%	successful transaction. Similary, a news/blog platform needs to have 
%	a trustworthy author/blogger. While censorship of information by editors 
%	are prevalent, they only reflect the personal opinion of consortium of 
%	people or organizations and in some cases may be biased. Every online 
%	platforms have their own reputation system 

	Online identities are an essential element in the process of digital 
	interaction and requires unknown entities to trust each other based on 
	reputation system of the platform in use. Let us take a simple
	scenario to present the context. Alice wants to buy a pair of 
	headphones. To achieve this, she browses the webshop, Dbay which is 
	a buy/sell platform. She then finds a relevant ad placed by Bob who 
	has good reviews. Alice picks up her credit card, submits required 
	details and waits for her desired product to be delivered as promised. 
	In this interaction, trustworthiness of Bob is fully reliant on Dbay. 
	The entity claiming to be Bob could be Eve who found a way to bypass 
	Dbay’s security and inflate his reputation on the system. Eve could 
	delete the ad and associated account when the payment is complete or 
	she could gather Alice’s personal details to misuse it later. Any 
	malformed decision on trustworthiness of an entity could be expensive 
	and deal severe damage to the user. A centrally governed system is more 
	prone to directed attacks. One successful attack on Dbay can leave 
	all its users vulnerable. Thus, a central point of authority is the 
	single point of failure.\\
	On the other hand information distributed over decentralized network 
	would require simultaneous attacks on numerous accounts to achieve 
	the same effect. Added layer of cryptographic security on this 
	network would hugely raise the difficulty level of attack. In 2008, 
	a solution was proposed as Bitcoin, a peer-to-peer electronic cash 
	system \cite{Bitcoin_Satoshi} that eliminated the need for trusted central 
	authority. The underlying technology, namely blockchain is a public 
	ledger that allows anyone on the network to audit blockchains state 
	changes and prove with mathematical certainity that transactions were 
	made according to the specified blockchain rule.
	\cite{enoughBitcoinForEthereum} It comes with a computationally expensive 
	proof of work for nodes to maintain and update the blockchain database. 
	Thus, blockchain ensures fault tolerance, zero downtime, tamper-resistant 
	data. Leveraging this technology for implementing reputation system
	could be an ideal solution. The use of right reputation algorithm 
	integrated with Blockchain can ensure trustworthiness of online 
	identities with high degree of certainity.	
	
	\section*{Task}
	The goal of this thesis is to use blockchain technology and smart contracts 
	to model a reputation system with graph theory and relevant reputation 
	algorithms in use. It aims to have a trust framework that can assist 
	decision making to either carry out the transaction or not.
	A reference network will be created where nodes and their relationships 
	will be studied to identify honest or malicious participants. The method to 
	calculate reputation score and infer trust value of associated identity 
	will be discussed and applied. Generalization of this reference network to 
	other transactional network to serve various other use cases will also be 
	discussed. 

	% reputation model will determine validity of transaction: if dishonest nodes
	% just endorsing each other with fake id's then the network flow algo will 
	% calculate net flow and realize that that node is malicious.
	Research questions that this project aims to address are : 
	\begin{enumerate}
		\item \textit{Reputation Model:} How can graph properties and similar 
			algorithms be used to formulate reputation model such that 
			malicious nodes are detected and trust transitivity is preserved 
			for infering the trust value for each nodes in the network? 
			Definition of the endorsement network, edges, direction, weight, 
			vertices.
			%and discouraged while honest nodes are incentivized to 
			%encourage clean behaviour in the network? 

		\item \textit{Blockchain:} What are the requirements for storing trust 
			values and linking them to associated identities stored off 
			blockchain network? How will the blockchain architecture look 
			like for endorsing known and unknown entities to serve decision 
			making process before two parties engage in any transaction? How can 
			blockchain application be built to define a general trust framework 
			for a transactional network? What should the overall system 
			architecture look like? 
		% Blockchain operations and its need.
		%\item How can blockchain application be built for a general trust 
		%	framework? What are the logic that needs to be considered  
		%	on a contract level? 

		\item \textit{UseCases:} How can the discussed endorsement network 
			ensure trustworthiness while also preserving users anonymity and 
			how can it be generalized to other transactional network or added 
			on top of it to serve other usecases such as content filtering, 
			E-Commerce etc? 
	\end{enumerate}

%	The task of this thesis project involves exploring current reputation 
%	model and trust frameworks used in various online systems. A blockchain 
%	based solution will be discussed and implemented as Proof-of-Concept for 
%	computing the trust value of interacting entities whose reputation scores 
%	are stored on public ledger for every nodes on the network to verify, 
%	validate and guarantee their trustworthiness. Discussion on various 
%	consensus algorithms for nodes to agree on a computational output will be
%	presented. Specifically, the transaction on the network would be the 
%	transfer of endorsement by known entities based on their physical or digital 
%	acquaintance. Verifying acquaintance and preventing sybil
%	attacks(i.e. attacker creates large number of pseudonymous identities 
%	to inflate their reputation or damage someone else’s) will be 
%	discussed as part of the reputation algorithm to be used. The aim
%	is to identify and investigate the problems and solutions of reputation 
%	model and trust frameworks used in current online systems.
	
	\paragraph{Main goal of the thesis project:}
	The Main goal is to prove that discussed trust framework works to evaluate 
	trustworthiness of participating entities via Proof of Concept(PoC).
%	The main goal is to design a PoC that can calculate reputation score and 
%	thus infer trust level of respective entities to aid in decision making 
%	before interaction takes place by using smartcontracts based decentralized 
%	reputation system.
%	 The goal is to design a PoC that can identify and protect 
%	 trustworthiness of interacting entities with higher degree of certianity 
%	 using smart contract based decentralized reputation system.	
	
	\section*{Objective to support the main goal: }
	\begin{itemize}
		\item \textit{Definition:} Identify relevant concepts, Current 
			reputation models, EigenTrust, Network flow convergence Algorithm.
		\item \textit{Analysis:} Evolution of blockchains, types, consensus 
			algorithms, other relevant blockchain concepts.
		\item \textit{Requirement Analysis:} Pre-requisite for test setup, 
			User stories, Determine user types, functions, functional and 
			non functional requirement for the system, identities, storage.
		\item \textit{Solution Design:} Formulate reputation as network flow 
			problem, use graph properties, EigenTrust.
		\item \textit{Implementation:} Write smartcontracts, deploy on ethereum 
			test network. 
		\item \textit{Result:} Measure, analyze, evaluate reputation scores 
			and trust accuracy.
		\item \textit{Documentation:} Write Report throughout the project 
			timeline. 
	\end{itemize}

	\section*{Method}
	Solidity, \cite{solidity} contract oriented programming language will be 
	used for writing contracts that define transactions and their exchange 
	methods on the peer-to-peer network of Ethereum. \cite{ethereum_whitepaper} 
	Test and deployment will be performed on Ethereum test network. Cloud 
	services(e.g.docker) may be used to test the validator nodes that follows 
	discussed consensus. For frontend to communicate with the contracts, 
	web3.js can be used. Git will be used as version control for the codes.
	%docker kubernetes plays well , hashicorp: terraform, nomad , Apache mesos	
	
	\section*{Relevant Literature}
	Some existing works carried out in relation to decentralized reputation 
	based on graph theories and various trust metrics are provided in 
	References section as:
	\cite{literature1} \cite{literature2} \cite{ilprints562}

	\section*{Relevant Courses}
	The relevant courses that the student has undertaken at Uppsala University, 
	are listed below. They are ordered based on their relevance to the project.
	\begin{enumerate}
		\item Cryptology, 5c
		\item Secure Computer Systems, 5c
		\item Applied Cloud Computing, 10c
		\item Advanced Software Design, 5c
		\item Programming Theory, 10c
		\item Algorithms and Data Structures II, 5c
	\end{enumerate}

	\section*{Delimitations}
	Given the time constraints, implementation will be limited to PoC for 
	endorsement network. Discussion on several usecases will be presented 
	but will not be experimented or tested with. Results will attempt to show 
	accuracy, failure probability but they will not be experimentally compared 
	to the existing systems for lack of time to search for appropriate datasets.
	If time permits, frontend may be developed such that contracts can be 
	interacted with from the web interface directly.
	
	\section*{Time Schedule}
	\vspace{1.5cm}
	\begin{center}
		\begin{tabular}{ |m{1.5cm}|m{30em} | } 
			\hline
			\textbf{Week} & \textbf{Description}\\
			\hline
			5 - 6 & Identify Relevant concepts: EigenTrust, Flow convergence \\
			7 - 8 & Evolution of blockchains, consensus algorithms \\
			9 - 10 & Model Network flow using graph properties \\
			11 - 13 & Setup test network, Write smartcontract code. \\
			12 & Midterm meeting \\
			14 - 16 & Results: Measure, Analyze \\
			17 - 18 & Write analysis of results \\
			19 - 20 & Write conclusion, futurework, complete final draft 
			for feedback \\
			21 - 22 & Prepare presentation, refactor code and documentation 
			for prototype \\
			23 & Backup time \\
			24 & Update with feedback, finalize paper, and present orally \\	
			\hline
		\end{tabular}
	\end{center}

	\setcounter{myWeekNum}{5}
    \ganttset{%
        calendar week text={\myWeek{}}%
    }
    \begin{figure}[h!bt]
        \begin{center}
            \begin{ganttchart}[
                vgrid={*{6}{draw=none}, dotted},
                hgrid={*{1}{draw=none}, dotted},
                x unit=.08cm,
                y unit title=.5cm,
                y unit chart=.75cm,
                newline shortcut=true,
                bar label node/.append style={align=justify},
                time slot format=isodate,
                time slot format/start date=2018-02-01]{2018-02-01}{2018-06-15}
                \ganttset{bar height=.6}
                \gantttitlecalendar{year, month=name, week} \\
                \ganttbar[bar/.append style={fill=BrickRed}]
				{Literature\ganttalignnewline Survey\ganttalignnewline }
                    {2018-02-01}{2018-02-09}\\

                \ganttbar[bar/.append style={fill=Blue}]{\textbf{1.} Writing}
                    {2018-02-05}{2018-05-18}\\

                \ganttbar[bar/.append style={fill=Cerulean}]{\textbf{1.1}
                Background}
                    {2018-02-12}{2018-02-23}\\

                \ganttbar[bar/.append style={fill=Emerald}]{\textbf{1.2}
                Method and \ganttalignnewline Solution Design}
                    {2018-03-01}{2018-03-09}\\
%
%                \ganttbar[bar/.append style={fill=SpringGreen}]{\textbf{1.3}
%                Solution Design}
%                    {2018-02-26}{2018-03-09}\\

                \ganttbar[bar/.append style={fill=Periwinkle}]{\textbf{1.3}
                Implementation Details}
                    {2018-03-16}{2018-03-30}\\

                \ganttbar[bar/.append style={fill=RedViolet}]{\textbf{1.4}
                Analysis \&  Evaluation}
                    {2018-04-27}{2018-05-04}\\

                \ganttbar[bar/.append style={fill=Dandelion}]{\textbf{1.5}
                Discussion \& Conclusion}
                    {2018-05-07}{2018-05-18}\\

                \ganttbar[bar/.append style={fill=SpringGreen}]
                {Design}
                    {2018-02-26}{2018-03-09}\\

                \ganttbar[bar/.append style={fill=Periwinkle
                }]{Implementation}{2018-03-12}{2018-03-30}\\

                \ganttbar[bar/.append style={fill=PineGreen}]
					{Testing}{2018-04-02}{2018-04-20}\\

%                \ganttbar[bar/.append style={fill=Violet}]
%					{Comparison}{2018-04-16}{2018-04-20}\\
%
                \ganttbar[bar/.append style={fill=RedViolet}]
				{Analysis}{2018-04-23}{2018-05-04}\\

                \ganttbar[bar/.append style={fill=Dandelion}]
					{Presentation}{2018-06-11}{2018-06-15}\\

            \end{ganttchart}
        \end{center}
        \caption{Time Plan}
    \end{figure}		
	\newpage
	\nocite{*}	
	\bibliographystyle{plain}
	\bibliography{specification}
	
\end{document}

