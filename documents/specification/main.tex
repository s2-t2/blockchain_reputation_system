\documentclass[a4paper,11pt,dvipsnames]{article}
\usepackage{geometry}
\geometry{
	a4paper,
	total={170mm,257mm},
	left=20mm,
	top=5mm,
}
\usepackage[linktocpage]{hyperref}
\usepackage[dvipsnames]{xcolor}

\usepackage{url}
\usepackage{hyperref} % should always be the last package

\usepackage[english]{babel}
\usepackage[utf8]{inputenc}
\usepackage{fancyhdr}
\usepackage{array}

%\usepackage{pgfgantt}
\usepackage[pdftex]{graphicx}   %Enable pdflatex
\usepackage{pgfgantt}

\newcounter{myWeekNum}
\stepcounter{myWeekNum}
%
\newcommand{\myWeek}{\themyWeekNum
	\stepcounter{myWeekNum}
	\ifnum\themyWeekNum=53
	\setcounter{myWeekNum}{1}
	\else\fi
}


\pagestyle{fancy}
\setlength{\headheight}{52pt}
\rhead{\today}

\begin{document}
	\begin{center}
		\textbf{\huge{Blockchain based Decentralized Reputation Model and 
		General Trust Framework}}
		\newline
%		\large Blockchain and SmartContract. \\
		\textsf{\large{Student: Sujata Tamang}}\\
		\textsf{\large{Supervisor: Jonatan Bergquist }}\\
		\textsf{\large{Reviewer: Björn Victor}}\\
		\textsf{\large{Department of Information Technology, Uppsala University}}
	\end{center}
	
	\section*{Background}
	%Define Reputation and Trust
	Online identities are an essential element in the process of digital 
	interaction and requires unknown entities to trust each other based on 
	the reputation system of platform in use. The reputation systems that are 
	widely used in today's online systems are application specific and each 
	service provider has their own rating system whereby users can leave 
	rating/feedback for a good or bad experience encountered.
	These models have various limitations and fail to provide a precise 
	accuracy range of user's trustworthiness.
	\cite{Josang:2007:STR:1225318.1225716} Any malformed decision on the 
	trustworthiness of an entity could be expensive and deal severe damage to 
	the user. A centrally governed system is more prone to directed attacks as 
	it has a single point of failure. 
	
	On the other hand, information distributed over a decentralized network would	require simultaneous attacks on numerous accounts to achieve the same effect.
	In 2008, a solution was proposed called Bitcoin, a peer-to-peer electronic 
	cash system \cite{Bitcoin_Satoshi} that eliminated the need for trusted 
	central authority. The underlying technology, namely blockchain, is a 
	public ledger that allows anyone on the network to audit blockchains state 
	changes and prove with mathematical certainty that transactions were made 
	according to the specified blockchain rule. \cite{enoughBitcoinForEthereum} 
	Different blockchain comes with various consensus algorithms based on the 
	application and use cases to maintain and update the blockchain database. 
	Blockchain ensures fault tolerance, zero downtime, and tamper resistant 
	data to be stored on its network. Leveraging this technology for 
	implementing reputation system could be an ideal solution. The use of 
	right reputation algorithm integrated with blockchain can ensure 
	trustworthiness of online identities with a high degree of certainty. 

	Previous work that has been done in context of reputation management 
	for online systems can be seen in \cite{literature1}\cite{literature2}
	\cite{COIN:COIN202}. EigenTrust \cite{ilprints562} is a reputation managemen 
	algorithm that is based on the notion of transitive trust(i.e. If a peer 
	\textit{i} trusts a peer \textit{j} then it implies that \textit{i} trusts 
	all other peers trusted by \textit{j}) and can be used on P2P network. 
	Bayesian network based trust model has been studied 
	by \cite{1231515} to build reputation based on recommendations in 
	P2P networks. Similarly, TrustDavis \cite{trustdavis} presents 
	an interesting method for non-exploitable online reputation system by 
	defining important characteristics of honest and malicious participants 
	and provides an incentive for accurate ratings of trust. 
%	Other existing works carried out in relation to decentralized reputation 
%	based on graph theories and various trust metrics can be seen in 
%	\cite{literature1} \cite{literature2}


%	EigenTrust is a reputation management algorithm for a P2P network that 
%	is based on the notion of transitive trust. \cite{ilprints562}. Similarly, 
	

%	Most reputation models base the trustworthiness of entity on difference 
%	of positive and negative feedbacks received in a given time frame.(cite) 
%	These methods have several problems and limitations. 

%	There is no standard definition of trust and reputation that is universally 
%	agreed upon. Therefore the definition varies depending upon the domain 
%	and context in use. For a peer-to-peer network, trust can be defined as a 
%	peer's belief in another peer's capabilities, honesty and reliability based 
%	on its own direct experiences whereas reputation is a peer's belief in 
%	another peer's capabilities, honesty and reliability based on
%	recommendations received from other peers. (cite) Most of the existing 
%	online applications calculate trust as difference in positive and negative 
%	ratings. (cite) Online identities are an essential element in the process 
%	of digital interaction and requires unknown entities to trust each other 
%	based on reputation system of the platform in use. The outcome of 
%	transaction is dependent on the accuracy range of respective reputation 
%	model. 
%	
%	Reputation system helps to 
%	determine the trustworthiness of an entity and predict the outcome of 
%	a transaction as success or failure beforehand. Many online systems 
%	use reputation system specific to their application and there is no 
%	general framework that fits all applications need. (cite) 
%
%
%	Several centrally governed system have reputation model that provide 
%	unreliable ratings and are often erroneous. As a central point of 
%	authority, its also a central point of failure so users need to rely on 
%	the trusted third party which is usually the online system admin or 
%	moderator. One could easily attack the system 
%	Use of EigenTrust has been studied for Reputation Management in Peer to 
%	peer network that calculates local and global trust of entities and 
%	trust transitivity. Graph theories has been used to study the property 
%	between nodes and analyze the peers interaction in the network. cite
%	However, there central system have their own limitation and decentralized 
%	reputation systems face various challenges when it comes to distinguishing 
%	between honest and malicious users and studying the interaction. Recent 
%	innovation of blockchain technology has been researched by various 
%	domains experts to experiment on different use cases, reputation being 
%	one. Blockchain based application such as steemit, akasha are already 
%	using reputation model built on blockchain that can filter content and 
%	rank them based on the peers voting. Sovrin is another blockchain based 
%	implementation that aims to provide ssi and decentralized trust by 
%	attesting real world physical identities. Blockchain ensures immutable 
%	data on a decentralized network that can be verified by anyone on the 
%	network. Consensus algorithms require everyone to agree on the block to 
%	be added to the chain. Proof of work requires a computationally expensive 
%	problem to be solved such that the transactions in the block can be 
%	written by miner who solved it. This ensures immutability and tamper 
%	resistance properties. Using Blockchain for reputation can greatly 
%	improve performance of reputation system and provide high degree of 
%	certainity for trustworthiness of entity. 

	\section*{Task}
	The goal of this thesis is to use blockchain technology and smart contracts 
	to model a reputation system with graph theory and relevant reputation 
	algorithms in use. It aims to have a trust framework that can assist 
	decision making to either carry out transaction or not.
	An endorsement network will be simulated where participants can endorse 
	each other based on physical or digital acquaintance. The nodes and 
	their relationships will be studied to identify honest or malicious 
	participants. The method to calculate reputation score and infer trust 
	value of associated identity will be discussed and applied. Generalization 
	of this endorsement network to other transactional networks to serve 
	various other use cases will also be discussed. 

	Research questions that this project aims to address are : 
	\begin{enumerate}
	  \item \textit{Reputation Model:} Definition of nodes, edges, direction, 
		  weight, vertices in the endorsement network. How can graph theories 
		  and relevant reputation algorithms be used to model the 
		  interaction between entities and detect/identify honest or malicious 
		  nodes? 
%		  reputation system and trust framework such that participating entities 
%		  can endorse each other based on physical or digital acquaintance that 
%		  can generate value in the network? 
		% graph - interaction between entities, interaction is endorsement transfer
		% detect and identify honest and malicious nodes.

		\item \textit{Blockchain:} What are the requirements for storing trust 
			values and linking them to associated identities stored off a
			blockchain network? How can a blockchain application be built to 
			define a general trust framework for a transactional network? How 
			could the overall system architecture look like? 

		% Blockchain operations and its need.
		%\item How can blockchain application be built for a general trust 
		%	framework? What are the logic that needs to be considered  
		%	on a contract level? 

		\item \textit{Use Cases:} How can the discussed endorsement network 
			ensure trustworthiness while also preserving users anonymity and 
			how can it be generalized to other transactional network or added 
			on top of it to serve other use cases such as content filtering, 
			E-Commerce etc? 
	\end{enumerate}
	
	\paragraph{Main goal of the thesis project:}
	The main goal is to design a PoC to simulate an endorsement network that 
	can compute reputation score and identify trustworthiness of interacting 
	entities using smart contract based decentralized reputation model. 


%	\section*{Objective to support the main goal: }
%	\begin{itemize}
%		\item \textit{Definition:} Identify relevant concepts, Current 
%			reputation models, EigenTrust, Network flow convergence Algorithm.
%		\item \textit{Analysis:} Evolution of blockchains, types, consensus 
%			algorithms, other relevant blockchain concepts.
%		\item \textit{Requirement Analysis:} Pre-requisite for test setup, 
%			User stories, Determine user types, functions, functional and 
%			non functional requirement for the system, identities, storage.
%		\item \textit{Solution Design:} Formulate reputation as network flow 
%			problem, use graph properties, EigenTrust.
%		\item \textit{Implementation:} Write smartcontracts, deploy on ethereum 
%			test network. 
%		\item \textit{Result:} Measure, analyze, evaluate reputation scores 
%			and trust accuracy.
%		\item \textit{Documentation:} Write Report throughout the project 
%			timeline. 
%	\end{itemize}

	\section*{Method}
	The methodology for this project will follow an incremental approach that 
	will start with definition of problem supported by literature review and 
	followed by solution design and implementation. The steps can be broken 
	down into: 
	\begin{itemize}
	  \item \textit{Definition:} Identify relevant concepts, algorithms, 
		current reputation models, evolution of blockchains, consensus 
		algorithms, other relevant blockchain concepts.
		%Read and write
	  \item \textit{Analysis:} pre-requisite for test setup, user stories, 
		determine user types, functions, functional and nonfunctional 
		requirements for the system, identities, on chain vs. off-chain storage. 
	  \item \textit{Solution Design:} Formulate reputation model using graph 
		properties, methods to quantify reputation score, trust values.
		%relevant axioms to be studied.
	  \item \textit{Implementation:} Write smart contract code, deploy on the 
		test network. 
	  \item \textit{Result:} Measure, analyze, evaluate reputation scores and 
		trust accuracy range.
	  \item \textit{Documentation:} Write report throughout the project timeline.
	\end{itemize}
	Solidity, \cite{solidity} a contract oriented programming language will be 
	used for writing contracts that define transactions and their exchange 
	methods on the peer-to-peer network of Ethereum. \cite{ethereum_whitepaper} 
	Test and deployment will be performed on Ethereum test network. Cloud 
	services such as docker may be used to test the validator nodes that follow 
	discussed consensus. Neo4j can be used for graph simulation.
	node.js can be used as a javascript run time and web3.js API may be used 
	to build a web application that can communicate with the smart contract. 
	Solc will be used to compile solidity code and Git will be used as version 
	control for the codes.
%	%docker kubernetes plays well , hashicorp: terraform, nomad , Apache mesos	
	
%	\section*{Relevant Literature}
%	Some existing works carried out in relation to decentralized reputation 
%	based on graph theories and various trust metrics are provided in 
%	References section as:
%	\cite{literature1} \cite{literature2} \cite{ilprints562}

	\section*{Relevant Courses}
	The relevant courses that the student has undertaken at Uppsala University, 
	are listed below. They are ordered based on their relevance to the project.
	\begin{enumerate}
		\item Cryptology, 5c
		\item Secure Computer Systems, 5c
		\item Applied Cloud Computing, 10c
		\item Advanced Software Design, 5c
		\item Programming Theory, 10c
		\item Algorithms and Data Structures II, 5c
	\end{enumerate}

	\section*{Delimitations}
	Given the time constraints, implementation will be limited to a PoC for 
	endorsement network. Discussion on several use cases will be presented 
	but will not be experimented or tested with. Results will attempt to show 
	reputation accuracy and failure probability but will not be experimentally 
	compared to the existing systems for lack of time to search for appropriate 
	datasets. If time permits, frontend may be developed such that contracts can 
	be interacted with from the web interface directly.
	
	\section*{Time Schedule}
	The week column in the time table corresponds to the respective week 
	shown in figure ~\ref{gantt}.
%	\vspace{1.5cm}
	\begin{center}
		\begin{tabular}{ |m{1.5cm}|m{30em} | } 
			\hline
			\textbf{Week} & \textbf{Description}\\
			\hline
			1 - 2 & Literature survey: Reputation algorithms, Identify relevant 
			concepts, Refine specification\\
			3 - 4 & Background: Evolution of blockchains, Consensus algorithms \\
			5 - 6 & Method \& Design: Model network using graph properties \\
			7 - 10 & Implementation: Setup test network, Write smart contract 
			code. \\
			10 & Midterm meeting \\
			11 - 12 & Testing: Measure, Analyze, evaluate \\
			13 - 14 & Analysis: Write analysis of results \\
			15 - 16 & Discussion \& Conclusion: Write conclusion, futurework, 
			complete final draft for feedback \\
			17 - 18 & Refactor code, documentation, prepare presentation \\
			%Prepare presentation, refactor code and documentation 
			%for prototype \\
			19 & Backup time \\
			20 & Presentation: Update with feedback, finalize paper, 
			and present orally \\	
			\hline
		\end{tabular}
	\end{center}

	\setcounter{myWeekNum}{1}
    \ganttset{%
        calendar week text={\myWeek{}}%
    }
    \begin{figure}[h!bt]
        \begin{center}
            \begin{ganttchart}[
                vgrid={*{6}{draw=none}, dotted},
                hgrid={*{1}{draw=none}, dotted},
                x unit=.07cm,
                y unit title=.5cm,
                y unit chart=.70cm,
                newline shortcut=true,
                bar label node/.append style={align=justify},
                time slot format=isodate,
                time slot format/start date=2018-02-01]{2018-02-01}{2018-06-15}
                \ganttset{bar height=.6}
                \gantttitlecalendar{year, month=name, week} \\
                \ganttbar[bar/.append style={fill=gray}]
				{Literature\ganttalignnewline Survey\ganttalignnewline }
                    {2018-02-01}{2018-02-09}\\

                \ganttbar[bar/.append style={fill=Blue}]{\textbf{1.} Writing}
                    {2018-02-05}{2018-05-18}\\

                \ganttbar[bar/.append style={fill=Cerulean}]{\textbf{1.1}
                Background}
                    {2018-02-12}{2018-02-23}\\

                \ganttbar[bar/.append style={fill=SpringGreen}]{\textbf{1.2}
                Method and \ganttalignnewline Solution Design}
                    {2018-03-01}{2018-03-09}\\
%
%                \ganttbar[bar/.append style={fill=SpringGreen}]{\textbf{1.3}
%                Solution Design}
%                    {2018-02-26}{2018-03-09}\\

                \ganttbar[bar/.append style={fill=Periwinkle}]{\textbf{1.3}
                Implementation Details}
                    {2018-03-18}{2018-04-06}\\

                \ganttbar[bar/.append style={fill=RedViolet}]{\textbf{1.4}
                Analysis \&  Evaluation}
                    {2018-04-27}{2018-05-04}\\

                \ganttbar[bar/.append style={fill=Dandelion}]{\textbf{1.5}
                Discussion \& Conclusion}
                    {2018-05-07}{2018-05-18}\\

                \ganttbar[bar/.append style={fill=SpringGreen}]
                {System Design}
                    {2018-02-26}{2018-03-09}\\

                \ganttbar[bar/.append style={fill=Periwinkle
                }]{Implementation}{2018-03-11}{2018-04-06}\\

                \ganttbar[bar/.append style={fill=PineGreen}]
					{Testing}{2018-04-08}{2018-04-20}\\

%                \ganttbar[bar/.append style={fill=Violet}]
%					{Comparison}{2018-04-16}{2018-04-20}\\
%
                \ganttbar[bar/.append style={fill=RedViolet}]
				{Analysis}{2018-04-23}{2018-05-04}\\


                \ganttbar[bar/.append style={fill=Dandelion}]
				{Refactor Code}{2018-05-20}{2018-05-31}\\

                \ganttbar[bar/.append style={fill=BrickRed}]
					{Presentation}{2018-06-11}{2018-06-15}
%
					\node (a) [fill=Gray,draw,anchor=west,minimum width=0.5cm] at ([yshift=-68pt]current bounding box.north east){};

				\node (b) [fill=Blue,draw,anchor=west, minimum width=0.5cm] at ([yshift=-10pt]a.south west){};
				\node (c) [fill=Cerulean,draw,anchor=west, minimum width=0.5cm] at ([yshift=-10pt]b.south west){};
				\node (d) [fill=SpringGreen,draw,anchor=west, minimum width=0.5cm] at ([yshift=-10pt]c.south west){};
				\node (e) [fill=Periwinkle,draw,anchor=west, minimum width=0.5cm] at ([yshift=-10pt]d.south west){};

				\node (f) [fill=PineGreen,draw,anchor=west, minimum width=0.5cm] at ([yshift=-10pt]e.south west){};

				\node (g) [fill=RedViolet,draw,anchor=west, minimum width=0.5cm] at ([yshift=-10pt]f.south west){};

				\node (h) [fill=Dandelion,draw,anchor=west, minimum width=0.5cm] at ([yshift=-10pt]g.south west){};

				\node (i) [fill=BrickRed,draw,anchor=west, minimum width=0.5cm] at ([yshift=-10pt]h.south west){};

				\node (j) [fill=White,anchor=west,minimum width=0.5cm] at ([yshift=-5pt]a.north east){Preliminary Reading};

				\node (k) [fill=White,anchor=west,minimum width=0.5cm] at ([yshift=-5pt]b.north east){Writing};

				\node (l) [fill=White,anchor=west,minimum width=0.5cm] at ([yshift=-5pt]c.north east){Preliminary Writing};

				\node (m) [fill=White,anchor=west,minimum width=0.5cm] at ([yshift=-5pt]d.north east){Design};

				\node (n) [fill=White,anchor=west,minimum width=0.5cm] at ([yshift=-5pt]e.north east){Implementation};


				\node (o) [fill=White,anchor=west,minimum width=0.5cm] at ([yshift=-5pt]f.north east){Test};

				\node (p) [fill=White,anchor=west,minimum width=0.5cm] at ([yshift=-5pt]g.north east){Analysis};


				\node (q) [fill=White,anchor=west,minimum width=0.5cm] at ([yshift=-5pt]h.north east){Conclusion};


				\node (q) [fill=White,anchor=west,minimum width=0.5cm] at ([yshift=-5pt]i.north east){Presentation};
            \end{ganttchart}
        \end{center}
        \caption{Time Plan}
		\label{gantt}
    \end{figure}		
	\newpage
	\nocite{*}	
	\bibliographystyle{plain}
	\bibliography{sample}
	
\end{document}

